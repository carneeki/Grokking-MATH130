%-----------------------------------------------------------------------------%
%- Algebra :: Exponentials & Logarithms --------------------------------------%
%-----------------------------------------------------------------------------%
\chapter{Exponentials \& Logarithms}
\label{sec:ExponentialsAndLogarithms}
Exponents, powers or indices and logarithms are ways of expressing numbers that
have been multiplied or divided a number of times.

If we were to plot an exponential equation on a graph, we would notice that
the graph has a constant doubling time, ie for every unit we double on the
$y$ axis, the number of units on the horizontal axis are constant.

If the graph increases at an increasing rate, it does not necessarily mean it
the graph is exponential.

\section{Powers, Exponentials and Indices}
\label{sec:PowersExponentialsAndIndices}

Indices are putting a power or index in the top right corner of a number,
which indicates how many times we must multiply or divide that number by
itself to give us a total number. With this information in mind, we need to
define some parts of the language behind how logs and indices should be
used.
\begin{align}
  {a}^{x} & = y
  \label{eq:ExponentialForm}
\end{align}
\begin{table}[!hbt]
\label{tab:PartsOfAnExponential}
\begin{tabularx}{\linewidth}{| l X |}
\hline
\multicolumn{2}{|l|}{Where:} \\
\hline \hline
a & is the base or \emph{radix}\\
x & is the index\\
y & is the output\\
\hline
\end{tabularx}
\caption{Parts of an exponential function}
\end{table}
\\
When we want to perform manipulations to several numbers, there are various
power laws we must bear in mind... They are summarised as follows:
\begin{align}
  {a}^{0}               & = 1 \label{eq:IndexLaw_Power0} \\
  {a}^{1}               & = a \label{eq:IndexLaw_Power1} \\
  {a}^{x} * {a}^{y}     & = {a}^{x+y} \label{eq:IndexLaw_AddExps} \\
  {a}^{x} - {a}^{y}     & = {a}^{x-y} \label{eq:IndexLaw_SubExps} \\
  {({a}^{x})}^{y}       & = {a}^{xy} \label{eq:IndexLaw_MultExps} \\
  a{b}^{-x}             & = \frac{a}{{b}^{x}} \label{eq:IndexLaw_NegExp} \\
  {a}^{(\frac{x}{y})}   & = {({a}^{x})}^{\frac{1}{y}}
                            \label{eq:IndexLaw_FracExp0} \\
                        & = \sqrt[y]{{a}^{x}} \label{eq:IndexLaw_FracExp1}\\
  a{b}^{(-\frac{x}{y})} & = \frac{a}{\sqrt[y]{{b}^{x}}}
\end{align}
Converting notation between the forms exhibited in
\ref{eq:IndexLaw_FracExp0} and \ref{eq:IndexLaw_FracExp1} is often
extremely useful in Calculus, in particular, differentiation.

\section{Logarithms}
\label{sec:Logarithms}
Until this point, only exponentials where we want to do something to a base
number have been covered. When we want to see how a base number has been
affected by it's power a logarithm is the way to undo the power.
\begin{quote}
  Think of a log as an undoing function to an exponential.\footnote{Chris
  Gordon, MATH130 lecturer for algebra stream, evening session on logarithms,
  2011.}
\end{quote}
Referring back to \ref{eq:ExponentialForm}:
\begin{align}
    {a}^{x} & = y \nonumber
\intertext{Now for the ``undoing function'':}
  \log_a(y) & = x
\end{align}
This gives rise to the easy to remember translation between the logarithms
and exponentials: ``logs are gay'' \footnote{Elizabeth Camilleri, advanced
mathematics student, on my whiteboard at home. Though, this really needs her
picture (included) to accompany the text for full effect.}
\begin{align}
  {a}^{x} = y & \equiv \log_a(y) = x \label{eq:LogsAreGay}
\end{align}
\begin{figure}[!htb]
  \centering
  \includegraphics[width=0.75\linewidth]{IMG_20110608_024826.jpg}
  \caption{LogsAreGay}
  \label{fig:LogsAreGay}
\end{figure}
The golden rule to remember when dealing with exponentials is the
Svensson-Cranbrook Log Method, to be sung to ``When you're happy and you know
it'':
\begin{quote}
  When you're having trouble with a power take a log.
\end{quote}

\subsection{Log Laws}
\label{sec:LogLaws}

\begin{align}
  log_b(1)           & = 0                         & \text{because} ~ & b^0              & = 1                    && \\
  log_b(b)           & = 1                         & \text{because} ~ & b^1              & = b                    && \\
  log_b(a)           & = \frac{log_d(a)}{log_d(b)} &                ~ &                  &                        && \\
  log_b(0)           & = \text{undefined}          & \text{because} ~ & b^x              & = 0 \text{is impossible} && \\
  log_b(xy)          & = log_b(x) + log_b(y)       & \text{because} ~ & b^c * b^d        & = b^{c+d}              && \\
  log_b(\frac{x}{y}) & = log_b(x) - log_b(y)       & \text{because} ~ & b^{c-d}          & = \frac{b^c}{b^d}      && \\
  log_b(x^d)         & = d * log_b(x)              & \text{because} ~ & (b^c)^d          & = b^{c*d}              && \\
  log_b(\sqrt[y]{x}) & = \frac{log_b(x)}{y}        & \text{because} ~ & \sqrt[y]{x}      & = x^{\frac{1}{y}}      && \\
  x^{log_b(y)}       & = y^{log_b(x)}              & \text{because} ~ & x^{log_b(y)}     & = b^{log_b(x)log_b(y)} && \\
                     &                             &                  &                  & = b^{log_b(y)log_b(x)} && \nonumber  \\
                     &                             &                  &                  & = y^{log_b(x)} \nonumber \\
  c * log_b(x) + d * log_b(y) & = log_b(x^c * y^d) & \text{because} ~ & log_b(x^c * y^d) & = log_b(y^d)           &&
\end{align}

\subsection{Converting Logarithm Bases}
Suppose there are two logarithms, $log_e(x)$

\subsection{Problem solving with logs}
Suppose we have to solve:
\begin{align}
  2^{x} & = \frac{5}{3^{x+1}} \\
\end{align}
We use logarithms:
\begin{align}
  log(2^x) & = log(\frac{5}{3^{x+1}}) \\
  x log(2) & = log(5) - log(3^{x+1}) \\
  x log(2) & = log(5) - (x+1) * log(3) \\
  x log(2) + x log(3)  & = log - log(3) \\
  x(log(2) + x log(3)) & = log(5) - log(3) \\
  x & = \frac{log(5) - log(3)}{log(2) + log(3)} \\
    & = \frac{log(\frac{5}{3})}{log(6)} \\
    & = \text{\ldots some decimal \ldots }
\end{align}

%-----------------------------------------------------------------------------%
%- Algebra :: Exponentials & Logarithms :: Euler Constant: Base e ------------%
%-----------------------------------------------------------------------------%
\section{Euler Constant: Base \emph{e}}
\label{sec:EulerConstantBaseE}
Most calculations involving logarithms will be to various bases depending on
the topic. In COMP it is often base 2 (binary), in other fields it is often base
10. In MATH and most of the sciences it is often base \emph{e}, which is
approximately 2.71828. \cite{duWGx}
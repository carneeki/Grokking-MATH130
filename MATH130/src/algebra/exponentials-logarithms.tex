%-----------------------------------------------------------------------------%
%- Algebra :: Exponentials & Logarithms --------------------------------------%
%-----------------------------------------------------------------------------%
\chapter{Exponentials \& Logarithms}
\label{chap:ExponentialsAndLogarithms}
Exponents, powers or indices and logarithms are ways of expressing numbers that
have been multiplied or divided a number of times.

If we were to plot an exponential equation on a graph, we would notice that
the graph has a constant doubling time, ie for every unit we double on the
$y$ axis, the number of units on the horizontal axis are constant.

If the graph increases at an increasing rate, it does not necessarily mean it
the graph is exponential.

\section{Powers, Exponentials and Indices}
\label{sec:PowersExponentialsAndIndices}

Indices are putting a power or index in the top right corner of a number,
which indicates how many times we must multiply or divide that number by
itself to give us a total number. With this information in mind, we need to
define some parts of the language behind how logs and indices should be
used.
\begin{align}
  {b}^{x} & = y \label{eq:IdxForm} \\
          & = b * b * \ldots * b \text{ ($x$ times)}
\end{align}
\begin{table}[!hbt]
\label{tab:PartsOfAnExponential}
\begin{tabularx}{\linewidth}{| l X |}
\hline
\multicolumn{2}{|l|}{The convention I will use here is:} \\
\hline \hline
b & is the base or \emph{radix}\\
x & is the index\\
y & is the output\\
\hline
\end{tabularx}
\caption{Parts of an exponential function}
\end{table}
\\
When we want to perform manipulations to several numbers, there are various
power laws we must bear in mind... They are summarised as follows:
\begin{align}
  {b}^{0}               & = 1 \label{eq:IndexLaw_Power0} \\
  {b}^{1}               & = b \label{eq:IndexLaw_Power1} \\
  {b}^{x} * {b}^{y}     & = {b}^{x+y} \label{eq:IndexLaw_AddIdxs} \\
  {b}^{x} - {b}^{y}     & = b^x * \frac{1}{b^y} \\
                        & = {b}^{x-y} \label{eq:IndexLaw_SubIdxs} \\
  {b}^{-(x+y)}          & = ({b}^{-1})^{x} * (a^{-1})^y \\
                        & = (b^{-x}) * (b^{-y}) \\
  {({b}^{x})}^{d}       & = {b}^{xd} \label{eq:IndexLaw_MultIdxs} \\
  b{x}^{-y}             & = \frac{b}{{x}^{y}} \label{eq:IndexLaw_NegIdx} \\
  \text{similarly} \nonumber \\
  b{x}^{-1}             & = \frac{1}{b} \label{eq:IndexLaw_Neg1} \\
  {b}^{(\frac{x}{y})}   & = {({b}^{x})}^{\frac{1}{y}}
                            \label{eq:IndexLaw_FracIdx0} \\
                        & = \sqrt[y]{{b}^{x}} \label{eq:IndexLaw_FracIdx1}\\
  d{b}^{(-\frac{x}{y})} & = \frac{d}{\sqrt[y]{{b}^{x}}} ~ \text{combining \eqref{eq:IndexLaw_NegIdx} and \eqref{eq:IndexLaw_FracIdx0}} \\
  \nonumber \\
  \text{Also note that }
  {b}^{x}               & = 0 ~ \text{is impossible} \label{eq:IndexLaw_EqualsZero}
\end{align}
Converting notation between the forms exhibited in
\ref{eq:IndexLaw_FracIdx0} and \ref{eq:IndexLaw_FracIdx1} is often
extremely useful in Calculus, in particular, differentiation.

\section{Logarithms}
\label{sec:Logarithms}
Until this point, only exponentials where we want to do something to a base
number have been covered. When we want to see how a base number has been
affected by it's power a logarithm is the way to undo the power.
\begin{quote}
  Think of a log as an undoing function to an exponential.\footnote{Chris
  Gordon, MATH130 lecturer for algebra stream, evening session on logarithms,
  2011.}
\end{quote}

\subsection{Log Laws}
\label{sec:LogLaws}
Many of these laws can be derived from the index laws, and have been included
in a way to preserve order with those laws for easier reference.
\begin{align}
  \log_b(1)           & = 0                     & \text{by \eqref{eq:IndexLaw_Power0}} \\
  \log_b(b)           & = 1                     & \text{by \eqref{eq:IndexLaw_Power1}} \\
  \log_b(xy)          & = \log_b(x) + \log_b(y) & \text{by \eqref{eq:IndexLaw_AddIdxs}} \\
  \log_b(\frac{x}{y}) & = \log_b(x) - \log_b(y) & \text{by \eqref{eq:IndexLaw_SubIdxs}} \\
  \log_b(x^d)         & = d * \log_b(x)         & \text{by \eqref{eq:IndexLaw_MultIdxs}} \\
  \log_b(\sqrt[y]{x}) & = \frac{\log_b(x)}{y}   & \text{by \eqref{eq:IndexLaw_FracIdx0}} \\
  \log_b(0)           & = \text{undefined}      & \text{by \eqref{eq:IndexLaw_EqualsZero}}
\end{align}
Here's where some new stuff is introduced:
\begin{align}
  b^{\log_b(x)}       & = x & \text{Logs of the same base cancel as an index}
  \label{eq:LogLaw_BaseCancel0} \\
  \log_b(b^x)         & = x & \text{Logs of the same base cancel as an index}
  \label{eq:LogLaw_BaseCancel1} \\
  \log_b(a)           & = ~ \frac{1}{\log_a(b)} & \text{Remember back to the
  initial statement\footnote{a log is an inverting function}}
  \label{eq:LogLaw_Inversion}
  \\
  \log_b(x)           & = \frac{\log_a(x)}{\log_a(b)} & \text{Changing log
  bases} \label{eq:LogLaw_ChangeBase} \\
  (\log_a(b))(\log_b(x)) & = \log_a(x)
\end{align}

Referring back to \ref{eq:IdxForm}, but replace the variable $b$ with $a$ gives
rise to the easy to remember translation between the logarithms
and exponentials: ``logs are gay'' \footnote{Elizabeth Camilleri, advanced
mathematics student, on my whiteboard at home. Though, this really needs her
picture (included) to accompany the text for full effect.}
\begin{align}
  \log_a(y) = x ~ \Longleftrightarrow ~ {a}^{x} = y  \label{eq:LogsAreGay}
\end{align}
\begin{figure}[!htb]
  \centering
  \includegraphics[width=0.75\linewidth]{IMG_20110608_024826.jpg}
  \caption{LogsAreGay}
  \label{fig:LogsAreGay}
\end{figure}

\subsection{Converting Logarithm Bases}
Suppose there are two logarithms, of different bases. It is often nice to use
logarithms of the same base as it keeps the maths simpler (and sometimes things
will divide or cancel out). To do this, we apply the change of base formula
$\eqref{eq:LogLaw_ChangeBase}$. Consider the following:
\begin{align}
  b^{log_b(x)} & = x \\
  log_a(b^{b * log_b(x)}) & = log_a(x) ~ \text{take log base-a of both sides} \\
  (log_a(b))(log_b(x)) & = \text{by \eqref{eq:IndexLaw_MultIdxs}}
\end{align}
There is an alternate way using division, and can be memorised as wrote:
\begin{align}
  log_b(x) & = \frac{log_a(x)}{log_a(b)} ~ \text{ \label{eq:LogLaw_ChangeBase} }
\end{align} 

\subsection{Problem solving with logs}
The golden rule to remember when dealing with indices and powers is the
Svensson-Cranbrookian Log Method, to be sung to ``When you're happy and you know
it'':
\begin{quote}
  When you're having trouble with a power take a log.
\end{quote}

Suppose we have to solve the following equation:
\begin{align}
  2^{x} & = \frac{5}{3^{x+1}}
\end{align}
We use logarithms:
\begin{align}
  2^{x} & = \frac{5}{3^{x+1}} \\
  log(2^x) & = log(\frac{5}{3^{x+1}}) \\
  x log(2) & = log(5) - log(3^{x+1}) \\
  x log(2) & = log(5) - (x+1) * log(3) \\
  x log(2) + x log(3)  & = log - log(3) \\
  x(log(2) + x log(3)) & = log(5) - log(3) \\
  x & = \frac{log(5) - log(3)}{log(2) + log(3)} \\
    & = \frac{log(\frac{5}{3})}{log(6)} \\
    & = \text{\ldots some decimal \ldots }
\end{align}

%-----------------------------------------------------------------------------%
%- Algebra :: Exponentials & Logarithms :: Euler Constant: Base e ------------%
%-----------------------------------------------------------------------------%
\section{Euler Constant: Base \emph{e}}
\label{sec:EulerConstantBaseE}
Most calculations involving logarithms will be to various bases depending on
the topic. In COMP it is often base 2 (binary), in other fields it is often base
10. In MATH and most of the sciences it is often base \emph{e}, which is
approximately 2.71828.\cite{duWGx} Numerically it is an interesting number as it
crops up all over the place, sometimes in the most bizarre of locations; however
we do not need to worry about that for MATH130.

The base $e$ is used on most calculators with a button that looks like $e^x$. To
use the logarithm with base $e$ it is usually a button that looks like ``ln'', or
$log_e(x)$.

An important thing to note about $e$, which will be raised further in calculus
chapters is when you differentiate $e$, it is the only function where it's
integral and derivative are both the same.   
%-----------------------------------------------------------------------------%
%- Algebra :: Polynomials ----------------------------------------------------%
%-----------------------------------------------------------------------------%  
\chapter{TODO: Polynomials}
\label{chap:Polynomials}
Definition: Sum of natural powers of $x$, scaled and added.

Example:
\begin{align}
  6x^4 + 20x^3 + \sqrt{2}x^2 + 13x - 2 \\
\end{align}

Two important terms are \emph{coefficient}; which is a number in front of the
$x$, and \emph{degree} which is the highest power of $x$ in the expression.

Polynomials are useful and common for approximating other functions. Here we
approximate sin(x):
\begin{align}
  &~ x - \frac{x^{3}}{6} \\
  &~ x - \frac{x^{3}}{6} + \frac{x^5}{120} \\
  &~ x - \frac{x^{3}}{3!} + \frac{x^5}{5!} \\
  &~ x - \frac{x^{3}}{3!} + \frac{x^5}{5!} - \frac{x^7}{7!} \\
sin(x) &~ = x - \frac{x^{3}}{3!} + \frac{x^5}{5!} - \frac{x^7}{7!} + \ldots \\
\end{align}

The main polynomials studied in MATH130 are:
\begin{description}
  \item[quadratics]
    \begin{align}
      y & = ax^2 + bx + c       && \text{General Form} \\
      y & = a(x -h)^{2} + k     && \text{Standard Form} \\
      y & = a(x - x_1)(x - x_2) && \text{Factored Form}
    \end{align}
  \item[cubics]
    \begin{align}
      y = ax^3 + bx^2 + cx + d \\
    \end{align}
\end{description}

\section{Quadratics}
The standard form is useful for finding the roots / zeros (where $y = 0$).
Consider solving:
\begin{align}
  8(x - 7)^{2} - 41 = 0 \\
\end{align}
Because there is only one $x$ value it makes it easier to solve:
\begin{align}
  8(x - 7)^{2} & = 41 \\
  (x - 7)^{2}  & = \frac{41}{8} \\
  x - 7        & = \pm \sqrt{\frac{41}{8}} \\
  x            & = \pm \sqrt{\frac{41}{8}} + 7
\end{align}

It is also useful for sketching. Consider:
\begin{align}
  y = 2(x-1)^{2} + 3
\end{align}
\begin{enumerate}
  %TODO: include graphs of each parabola in this enumeration
  \item Start with $y = x^2$
  
  \item Because there is an $(x - 1)$ we shift the parabola $y = x^2$ by 1 to
  the right.\\
  $y = (x - 1)^2$
  
  \item Because there is a $2$ in front of the $(x-1)$, we make the parabola
  steeper by a factor of two.\footnote{called ``stretching vertically''} \\
  $y = 2(x - 1)^2$
  
  \item Because there is a $+3$ on the end, it lifts the parabola up by 3. \\
  $y = 2(x - 1)^2 +3$
\end{enumerate}


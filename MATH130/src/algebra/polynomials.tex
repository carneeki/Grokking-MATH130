%-----------------------------------------------------------------------------%
%- Algebra :: Polynomials ----------------------------------------------------%
%-----------------------------------------------------------------------------%  
\chapter{TODO: Polynomials}
\label{chap:Polynomials}
Definition: Sum of natural powers of $x$, scaled and added.

Example:
\begin{align}
  6x^4 + 20x^3 + \sqrt{2}x^2 + 13x - 2 \\
\end{align}

Two important terms are \emph{coefficient}; which is a number in front of the
$x$, and \emph{degree} which is the highest power of $x$ in the expression.

Polynomials are useful and common for approximating other functions. Here we
approximate sin(x):
\begin{align}
  &~ x - \frac{x^{3}}{6} \\
  &~ x - \frac{x^{3}}{6} + \frac{x^5}{120} \\
  &~ x - \frac{x^{3}}{3!} + \frac{x^5}{5!} \\
  &~ x - \frac{x^{3}}{3!} + \frac{x^5}{5!} - \frac{x^7}{7!} \\
  &~ x - \frac{x^{3}}{3!} + \frac{x^5}{5!} - \frac{x^7}{7!} + \frac{x^9}{9!} \\
  &~ x - \frac{x^{3}}{3!} + \frac{x^5}{5!} - \frac{x^7}{7!} + \frac{x^9}{9!} -
  \frac{x^11}{11!} \\
 \sin(x) \approx &~ x - \frac{x^{3}}{3!} + \frac{x^5}{5!} - \frac{x^7}{7!} +
 \frac{x^9}{9!} - \frac{x^11}{11!} \\
\end{align}

The main polynomials studied in MATH130 are:
\begin{description}
  \item[quadratics]
    \begin{align}
      y & = ax^2 + bx + c       && \text{General Form} \\
      y & = a(x - h)^{2} + k    && \text{Standard Form} \\
      y & = a(x - x_1)(x - x_2) && \text{Factored Form}
    \end{align}
  \item[cubics]
    \begin{align}
      y = ax^3 + bx^2 + cx + d
    \end{align}
\end{description}

\section{Quadratics}
\subsection{General Form}
\begin{align}
  y & = ax^2 + bx + c \nonumber
\end{align}
This form is perhaps the most common form of an equation - it is completely
unfactored and appears the most natural (at least to me).
\\
Key advantages of general form:
\begin{itemize}
  \item $c = y$-intercept and is easily read straight off the equation.
  \item $a$ tells us a lot about the shape of the parabola:
  \begin{itemize}
      \item A ``happy'' parabola has a positive value
      \item A ``sad'' parabola has a negative value
      \item Larger values of $a$ give steeper parabolas
  \end{itemize}
\end{itemize}

\subsection{Standard Form}
\begin{align}
  y & = a(x - h)^{2} + k \nonumber
\end{align}
Key advantages of standard form (sometimes called the \emph{vertex form}):
\begin{itemize}
  \item $a$ tells us the same information about the shape of the parabolas as in
  the general form.
  \item $h$ tells us the $x$ coordinate of the minimum/maximum point, however,
  to get the coordinate, we must multiply $h$ by $-1$.
  \item $k$ tells us the $y$ coordinate of the minimum/maximum point.
  \item Useful for finding the roots / zeros (where $y = 0$) of a quadratic
\end{itemize}
Consider solving:
\begin{align}
  8(x - 7)^{2} - 41 = 0
\end{align}
Because there is only one $x$ value it makes it easier to solve:
\begin{align}
  8(x - 7)^{2} & = 41 \\
  (x - 7)^{2}  & = \frac{41}{8} \\
  x - 7        & = \pm \sqrt{\frac{41}{8}} \\
  x            & = \pm \sqrt{\frac{41}{8}} + 7
\end{align}

It is also useful for sketching. Consider:
\begin{align}
  y = 2(x-1)^{2} + 3
\end{align}
\begin{enumerate}
  %TODO: include graphs of each parabola in this enumeration
  \item Start with $y = x^2$
  
  \item Because there is an $(x - 1)$ we shift the parabola $y = x^2$ by 1 to
  the right.\\
  $y = (x - 1)^2$
  
  \item Because there is a $2$ in front of the $(x-1)$, we make the parabola
  steeper by a factor of two.\footnote{called ``stretching vertically''} \\
  $y = 2(x - 1)^2$
  
  \item Because there is a $+3$ on the end, it lifts the parabola up by 3. \\
  $y = 2(x - 1)^2 +3$
\end{enumerate}

\subsection{Factored Form}
\begin{align}
  y & = a(x - x_1)(x - x_2) \nonumber
\end{align}
Key advantages of standard form:
\begin{itemize}
  \item $a$ tells us the same information about the shape of the parabolas as in
  the general form.
  \item $x_1$ and $x_2$ give us the $x$-intercepts (when multiplied by $-1$) of
  the equation.
\end{itemize}
Factored form is useful for finding the equation given a parabola. Suppose we
know the $x$-intercepts of a parabola are $-3$ and $-1$, and it has a
$y$-intercept of 6:
\begin{align}
  f(x) & = a(x - x_1)(x - x_2) \\
       & = a(x - -3)(x - -1) \\
       & = a(x +3)(x + 1) \\
     6 & = a(0 + 3)(0 + 1) \text{substitute $x=0$ to get $y$-int} \\
     6 & = a(3) \\
     2 & = a \\
     \therefore f(x) & = 2(x+3)(x+1) \\
     & = 2x^2 + 8x + 6 
\end{align}

\subsection{Quadratic Formula}
To get the roots of a quadratic (to factorize it), there is a
long\footnote{horrible} formula we can use called the \emph{quadratic formula}
to get values of $x$. If given the general form of the quadratic, we can
substitute the values of $a$, $b$, and $c$ into this formula to get values for
$x$ which can then be used in the factored form.
\begin{align}
  x = \frac{-b \pm \sqrt{b^2 - 4ac}}{2a}
\end{align}
By applying the quadratic formula to our previous example from the factored
form, we can show all 3 forms of the quadratic:
\begin{align}
  & = 2x^2 + 8x + 6 \nonumber \\
  \text{where} & a = 2, b = 8, c = 6 \nonumber \\
  x & = \frac{-b \pm \sqrt{b^2 - 4ac}}{2a} \\
    & = \frac{-8 \pm \sqrt{8^2 - 4(2)(6}}{2(2)} \\
    & = \frac{-8 \pm \sqrt{56 - 24}}{4} \\
    & = \frac{-8 \pm \sqrt{32}}{4} \\
    & = \frac{-8 \pm 4\sqrt{2}}{4}) \\    
    & = \pm 8\sqrt{2} \\
    \text{substitute $x$ into vertex form } \\
    f(x) & = a()^2 + k
\end{align}

\section{Cubics}
\lipsum[1]
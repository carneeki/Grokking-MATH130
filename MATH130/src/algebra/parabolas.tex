%-----------------------------------------------------------------------------%
%- Algebra :: Parabolas ------------------------------------------------------%
%-----------------------------------------------------------------------------%
\chapter{Parabolas}
\label{chap:Parabolas}
\begin{figure}[!hbt]
\label{fig:FuncGraphParabola}
\begin{tikzpicture}[scale=1]
  \def\xmin{-5}
  \def\xmax{5}
  \def\ymin{-2}
  \def\ymax{5}
  \clip
    (\xmin,\ymin) rectangle (\xmax,\ymax);
  \filldraw[fill=blue,fill opacity=0.15, color=neekiBlue, <->, variable=x]
    plot[id=parabola, raw gnuplot, smooth]
    function{
      set xrange [\xmin:\xmax];
      set yrange [\ymin:\ymax];
      plot x**2;}
    node[right] {$f(x) = {x}^{2}, x \in \{ -5,5 \} $};
  % grid
  \draw[very thin, color=black!30, ystep=1, xstep=1]
    (\xmin,\ymin) grid (\xmax,\ymax);
  % x-axis
  \draw[<->]
    (\xmin,0) -- (\xmax,0) node[right] {$x$};
  % y-axis
  \draw[<->]
    (0,\ymin) -- (0,\ymax) node[above] {$f(x)$};
\end{tikzpicture}
\caption{A parabolic function: $f(x) = {x}^{2}$}
\end{figure}
A parabolic function in the form $f(x) = mx^2 + c$
\begin{table}[!hbt]
\label{tab:PartsOfAParabolicFunction}
\begin{tabularx}{\linewidth}{| l X |}
  \hline
  \multicolumn{2}{|l|}{Where:} \\
  \hline \hline
  m & represents a component of the gradient (covered more in section
  \ref{sec:Differentiation} ``Differentiation''). \\ x & is the independent
  variable\\ y & is the dependent variable\\
\hline
\end{tabularx}
\end{table}
%-----------------------------------------------------------------------------%
%- Algebra :: Parabolas ------------------------------------------------------%
%-----------------------------------------------------------------------------%
\chapter{Parabolas}
\label{chap:Parabolas}
\begin{figure}[!hbt]
\label{fig:FuncGraphParabola}
\begin{tikzpicture}[domain=-5:5]
  \draw [color=neekiGrey,dash pattern=on 1pt off 1pt,xstep=1cm,ystep=1cm]
  (-5,-5) grid (5,5); \draw[->,color=black] (-5,0) -- (5,0);
  \foreach \x in {-5,-4,-3,-2,-1,0,1,2,3,4,5}
    \draw[shift={(\x,0)},color=black] (0pt,2pt) -- (0pt,-2pt) node[below]
    {\footnotesize $\x$};
  \draw[->,color=black] (0,-5) -- (0,5);
  \foreach \y in {-5,-4,-3,-2,-1,0,1,2,3,4,5}
    \draw[shift={(0,\y)},color=black] (2pt,0pt) -- (-2pt,0pt) node[left]
    {\footnotesize $\y$};
  \clip(-5,-5) rectangle (5,5);
  \draw[color=neekiBlue,samples=100,thick]
    plot[id=parabolaxsquared] function{x**2}
    node[right] at (1.5,1.5) {$f(x)=x^2$};
  \draw[color=neekiRed,samples=100,thick]
    plot[id=parabola-xsquared] function{-x**2}
    node[left] at (-1.5,-1.5) {$f(x)=-x^2$};
\end{tikzpicture}
\caption{A parabolic function: $f(x) = {x}^{2}$}
\end{figure}
\begin{table}[!hbt]
\label{tab:PartsOfAParabolicFunction}
\begin{tabularx}{\linewidth}{| l X |}
  \hline
  \multicolumn{2}{|l|}{Where:} \\
  \hline \hline
  m & represents a component of the gradient (covered more in section
  \ref{sec:Differentiation} ``Differentiation''). \\ x & is the independent
  variable\\ y & is the dependent variable\\
\hline
\end{tabularx}
\end{table}
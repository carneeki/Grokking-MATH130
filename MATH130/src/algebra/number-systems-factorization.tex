%-----------------------------------------------------------------------------%
%- Algebra :: Number Systems & Factorization ---------------------------------%
%-----------------------------------------------------------------------------%
\chapter{Number Systems \& Factorization}
\label{sec:NumberSystemsAndFactorization}
There are some basic laws that need to be understood to manipulate numbers
The first group of laws are called the "Distributive laws".
\begin{align}
      a(b+c) & = ab + ac \label{eq:distrib0} \\
      (a+b)c & = ac + bc \label{eq:distrib1} \\
  (a+b)(c+d) & = ac + ad + bc + bd \label{eq:distrib2}
\end{align}
Equation \ref{eq:distrib2} gives rise to a special case called a quadratic
which will be introduced in section \ref{sec:IntroductionToPolynomials},
Introduction to Polynomials, and in further detail in chapter
\ref{chap:Polynomials}, Polynomials. 
\\
The distributive laws are all about expanding brackets, that is to say, in
\ref{eq:distrib1}, first we multiply $a$ with the first term inside the
brackets ($b$) to give us $ab$, then we multply $a$ with the second term, $c$
to give us $ac$. When we add them (the $+$ symbol in the brackets) we get
$ab + ac$.
\\
Another way to think about it is $a$ is distributed to each term inside the
brackets. This also applies in \ref{eq:distrib2} with $c$, and it yields the 
same result as in \ref{eq:distrib1}.

%-----------------------------------------------------------------------------%
%- Algebra :: Number Systems & Factorization :: Fractions --------------------%
%-----------------------------------------------------------------------------%
\section{Introduction to Fractions}
\label{sec:IntroductionToFractions}
It turns out that any repeating decimal can be written as a fraction:

\begin{align}
  2.1 \times 100000 & = \\
    & = 2.1 \times \frac{10}{10} \\
    & = \frac{21}{10} \\
\end{align}

What about $1.33333\ldots$?
\begin{align}
  1.33333 & = \frac{4}{3} \\
\end{align}

Or $1.373737\ldots$ ??
\begin{align}
  let     x & = 1.373737\ldots\\
       100x & = 137.3737\ldots\\
   100x - x & = 137.373737\ldots - 1.373737\ldots\\
        99x & = 136 \\
  so \nonumber \\
        x & = \frac{136}{99} \\
        \therefore 1.373737\ldots = x & = \frac{136}{99}  
\end{align}

For longer decimal places we need to create 2 numbers using $x$ with the same
decimal part so that when we subtract the decimal part we get a whole number.

\begin{align}
         let  x & = 36.2593593593\ldots \\
       so 10x & = 362.593593\ldots \\
       10000x & = 362593.593\ldots \\
   10000x-10x & = 362593.593\ldots - 362.593\ldots \\
        9990x & = \ldots
 \intertext{(we have now subtracted the recurring decimal component
 from the fraction)}
 \therefore x & = \frac{362593 - 362}{9990} \\
\end{align}
As it turns out, fractions will fall into one of three categories:
\begin{enumerate}
  \item recurring decimals such as $\frac{99}{101} = 0.9801\ldots$.
  \item non-recurring decimals such as $\frac{1}{2} = 0.5$, these first two are
  called $\mathbb{Q}$ or rational numbers.
  \item non-recurring decimals such as $\pi = 3.1415\ldots$ or $\sqrt{2} =
  1.4142\ldots$, these are denoted by the symbol $\mathbb{I}$, and called
  irrational numbers\footnote{These numbers require some complex calculus to
  prove they are non-recurring, a simpler number is
  $0.1234567890111121314\ldots$ which has a pattern, and is not recurring, it's
  just ``add one'' to the last number}.
\end{enumerate}

%-----------------------------------------------------------------------------%
%- Algebra :: Number Systems & Factorization :: Fractional Operations --------%
%-----------------------------------------------------------------------------%
\subsection{Fractional Operations - Adding}
\label{sec:FractionalOperationsAdding}
A good reason why we \emph{don't} add fractions in the following way:
\quotation{add the tops (to give the numerator), add the bottoms (to give the
denominator (or quotient)}.
example: $\frac{1}{2} + \frac{1}{2} \neq \frac{2}{4} = \frac{1}{2}$

The \emph{correct} way of adding fractions is to use a common quotient then add
the numerator and keep the denominator the same.\footnote{a simpler (but only
partial) answer could be ``you're adding like with like'' -- a MATH130 student
from the audience of Chris Gordon's lecture at 2011-08-08 10:31AM}

%% TODO: insert a pizza divided into 8 slices and add 3 + 2 slices to represent
% part of the pizza.
Consider you order 3 slices of pizza from Hot Momma's pizza at the MQ bar, and
you are given 2 more for being a regular customer:
\begin{align}
  \frac{3}{8} & = \ldots \\
  \frac{3}{8} + \frac{2}{8} & = \frac{5}{8} \\
\end{align}
You have $\frac{5}{8}$ or ``five eights'' of a whole pizza.\footnote{I want a
slice of that pizza if it's the supreme}.

What if you have different quotients? You \emph{need} to convert to a common
quotient:
\begin{align}
 \frac{1}{2} + \frac{7}{10} & = \\
    & = \frac{5}{5} \times \frac{1}{2} + \frac{7}{10} \\
    & = \frac{5}{10} \times \frac{7}{10} \label{eq:CommonDenominator}\\
    & = \frac{12}{10} \\
    & = \frac{2 \times 6}{2 \times 5} \\
  \intertext{the two's divide out, which simplifies the fraction}  
    & = \frac{6}{5} \\
\end{align}
Equation \ref{eq:CommonDenominator} is where the important heavy lifting of the
operation of converting to a common quotient comes into play.

A general case:

\begin{align}
  \frac{a}{b} + \frac{c}{d} & = \\
    & = [\frac{a}{b} \ times \frac{d}{d}] + [\frac{c}{d} \ times \frac{b}{b} ]
    & = \frac{ad + cd}{db} \\
\end{align}

%-----------------------------------------------------------------------------%
%- Algebra :: Number Systems & Factorization :: Fractional Operations --------%
%-----------------------------------------------------------------------------%
\subsection{Fractional Operations - Division}
\label{sec:FractionalOperationsDivision}
Dividing fractions has a reasonably simply rule to remember: Multiply by the
inverse of one fraction:

\begin{align}
  \frac{a}{b} \div \frac{c}{d} & = \frac{ \frac{a}{b} }{ \frac{c}{d}} \times
  \frac{bd}{bd} \\
   & = \frac{\frac{a}{b}\times b \times d}{\frac{c}{d} \ times b \times d} \\
  \intertext{divide out common terms}
   & = \frac{ad}{cb}
\end{align}

%-----------------------------------------------------------------------------%
%- Algebra :: Number Systems & Factorization :: Surds ------------------------%
%-----------------------------------------------------------------------------%
\section{Introduction to Irrational Numbers (Surds)}
\label{sec:IntroductionToIrrationalNumbers}
A \emph{surd} is \emph{an archaic\footnote{almost as old as the Maths
Deptartment} term for an irrational number}. This is basically a number which
cannot be written as a decimal because, if you tried
\begin{enumerate}
  \item you'd go on forever as it has an infinite number of decimal places\\
  and
  \item it has no repeating parts to the decimal places.
\end{enumerate}
For this reason it cannot be expressed as a fraction in the form $\frac{p}{q}$
where p and q are integers.
Examples of irrational numbers are $e, \pi, \sqrt[2]{2}$
\\
A more formal\footnote{poorly worded, but means the same thing as above}
definition is:
\begin{equation*}
\left.\begin{aligned}
  \mathbb{I} \ni \frac{p}{q}
\end{aligned}
\right\} 
\qquad \text{{p,q $\in \mathbb{Z}$}}
\end{equation*}
\\
There are some handy things we can do with irrational numbers. Consider
$\sqrt[2]{8} =
2.828427124$\footnote{It's even longer than
2.8284271247461900976033774484193961571393437507538961463533594759814649...,
it's infinite remember!}. It can be rewritten like this:
\begin{align}
  \sqrt[2]{8} & = 
    \sqrt[2]{2 \times 4} \\
    & = \sqrt[2]{2} \times \sqrt[2]{4} \\
    & = \sqrt[2]{2} \times 2 \\
    & = 2\sqrt[2]{2}
\end{align}

\subsection{Rationalizing the Denominator}
\label{sec:RationalizingTheDenominator}
Often examiners will give us a fraction and say ``rationalize the
denominator''\footnote{``sudo rationalize the denominator'' if you want to be
a troll}. I don't know why - they just do. In order to get the marks in the
exam, we can rationalize the denominator by multiplying that fraction by 1.
\\
While the notion of multiplying by 1 sounds silly, consider that $1 \in
\mathbb{R} = \frac{p}{q}$ where \{p,q\} can be a surd, $x$: $\frac{x}{x} =
1$. This gives rise to the following possibility of:
\begin{align}
  \frac{5}{\sqrt[2]{2}} & = \\
   & = \frac{5}{\sqrt[2]{2}} \times \frac{\sqrt[2]{2}}{\sqrt[2]{2}} \\
   \intertext{The next part is where the useful stuff happens, if we square a
   square-root then they ``undo'' each other, and we are left over with the bit
   inside the square-root}
   & = \frac{5 \times \sqrt[2]{2}}{(\sqrt[2]{2}) \times (\sqrt[2]{2})} \\
   & = \frac{5\sqrt[2]{2}}{2}
\end{align}
The denominator might not always be a square-root, University of North Texas'
next example includes a cube-root, so we must multiply by 1 again.
\begin{align}
  \frac{2}{\sqrt[3]{5}} & = \\
    \intertext{sometimes it's nicer to lay things out to see what's going on:}
    & = \frac{2}{\sqrt[3]{5}} \times
          \frac{\sqrt[3]{5}}{\sqrt[3]{5}} \times
          \frac{\sqrt[3]{5}}{\sqrt[3]{5}}
    \intertext{but we still condense it into the root symbol:}
    & = \frac{2}{\sqrt[3]{5}} \times
          \frac{\sqrt[3]{5 \times 5}}{\sqrt[3]{5 \times 5}} \\
    & = \frac{2 \sqrt[3]{5 \times 5}}{(\sqrt[3]{5})(\sqrt[3]{5 \times
          5})} \\ & = \frac{2 \sqrt[3]{25}}{5} \\
\end{align}
The last example involves more than one term on the
denominator. In this particular case, we will still multiply by 1, however we
are using some trickery from an upcoming section
\ref{sec:IntroductionToPolynomials}, ``Introduction to Polynomials''
specifically equation \ref{eq:Diff2Squares}.
\begin{align}
  \frac{2}{1+\sqrt[2]{3}} & = \\ 
  & = \frac{2}{1+\sqrt[2]{3}} \times \frac{1-\sqrt[2]{3}}{1-\sqrt[2]{3}} \\
  & = \frac{2(1-\sqrt[2]{3})}{1-3}
  \intertext{if the denominator part of above step does not make sense, then
  please refer to equation \ref{eq:Diff2Squares} in section
  \ref{sec:IntroductionToPolynomials}, ``Introduction to Polynomials''}
  & = \frac{2(1-\sqrt[2]{3})}{-2} \\
  & = -\frac{2(1-\sqrt[2]{3})}{2} \\
  & = -\frac{1(1-\sqrt[2]{3})}{1} \\
  & = -(1-\sqrt[2]{3})
\end{align}
The examples for rationalizing the denominator come from University of Northern
Texas:
\url{http://www.math.unt.edu/mathlab/emathlab/How\%20to\%20Rationalize\%20the\%20Denominator\%20of\%20a\%20Fraction.htm}

%-----------------------------------------------------------------------------%
%- Algebra :: Number Systems & Factorization :: Polynomials ------------------%
%-----------------------------------------------------------------------------%
\section{Introduction to Polynomials}
\label{sec:IntroductionToPolynomials}
This section forms only an introduction to polynomials. More detail on
polynomials is covered in chapter \ref{chap:Polynomials}, ``Polynomials''.

Polynomials are a way of packing certain types of long equations into neater,
more compact forms. The following equations show how the distributive laws can
be applied to 3 polynomial equations. These 3 equations form the basic 3 rules
of polynomials and their form should be memorised to make solving more complex
problems easier down the track.
\begin{align}
  {(a+b)}^{2} & = & (a+b)(a+b) \nonumber \\
              & = & {a}^{2} + 2ab + {b}^{2} \label{eq:poly0} \\
  {(a-b)}^{2} & = & (a-b)(a-b) \nonumber \\
              & = & {a}^{2} - 2ab + {b}^{2} \label{eq:poly1} \\
  (a+b)(a-b)  & = & {a}^{3} + ab - ab - {b}^{2} \nonumber \\ 
              & = & {a}^{2} - {b}^{2} \label{eq:Diff2Squares} \\
  (a+b)({a}^{2} - ab + {b}^{2}) & = & {a}^{3} + {b}^{3} \label{eq:Sum2Cubes} \\
  (a-b)({a}^{2} + ab + {b}^{2}) & = & {a}^{3} - {b}^{3} \label{eq:Diff2Cubes}
\end{align}
Equation \ref{eq:Diff2Squares} is often called the difference of two squares where
${a}^{2}$ and ${b}^{2}$ represent both squares. \\
Equation \ref{eq:Diff2Cubes} is often called the difference of two cubes.
%-----------------------------------------------------------------------------%
%- Algebra :: Number Systems & Factorization :: Quadratics -------------------%
%-----------------------------------------------------------------------------%
\section{Quadratics}
\label{sec:Quadratics}
Quadratics are an important type of the distributive law. They represent 3
coefficients and a variable. Equations \ref{eq:poly0} through to \ref{eq:Diff2Squares}
are the classic 3 ways in which quadratics are introduced in textbooks. A more
formal definition has been provided by the table
\ref{eq:ComponentsToAQuadratic} (from \cite{MD51J}).
\begin{equation}
  a{x}^{2} + bx + c = 0
  \label{eq:ComponentsToAQuadratic}
\end{equation}
\begin{table}[!htb]
\begin{tabularx}{\linewidth}{| l X |}
\hline
\multicolumn{2}{|l|}{Where:} \\
\hline \hline
x & is the indeterminate variable \\
a & is the quadratic coefficient \\
b & is the linear coefficient \\
c & is the constant coefficient \\
\hline
\end{tabularx}
\caption{Components to a quadratic}
\end{table}
These particular quadratics are often called squares which are covered in more
detail in section \ref{sec:CompletingTheSquare}, Completing the Square.
%-----------------------------------------------------------------------------%
%- Algebra :: Number Systems & Factorization :: Completing the Square --------%
%-----------------------------------------------------------------------------%
\newpage
\section{Completing the Square}
\label{sec:CompletingTheSquare}
Completing the square is useful for solving quadratic equations as well
as graphing quadratic functions, as well as evaluating integrals in calculus.
The key concept behind completing the square in MATH130 is that we want to
convert a quadratic polynomial like:
\begin{align}
  a{x}^{2} + bx + c \nonumber
  \intertext{into the form}
  a{(x-h)}^{2} + k \nonumber
\end{align}
To do this we must find $h$ and $k$. There are two ways about doing this
depending on whether the value of $a = 1 $.
\subsection{General Case, when a = 1}
If given an equation like:
\begin{align}
{x}^{2} + bx + c &
\intertext{we can form a square like this:}
{(x+\frac{1}{2}b)}^{2} & = {x}^{2} + bx + \frac{1}{4} {b}^{2}
\intertext{however, we have not taken into account the constant $c$ so, we
should really write the following to take it into account}
{x}^{2} + bx + c & = {(x+\frac{1}{2}b)}^{2} + k
\end{align}
The following examples come courtesy of Wikipedia \cite{UmibR} (with
intermediary working provided by author):
\begin{align}
        {x}^{2} + 6x + 11 & =
  \intertext{First: halve $b$ to give us $3$, and then put it into the complete
  square form (remembering the k-value):}
                          & = {(x+3)}^{2} + k
  \intertext{Next calculate $k$}
              {(x+3)}^{2} & = {x}^{2} + 6x + 9 \\
                   11 - 9 & = 2 \\
               \therefore k & = 2
  \intertext{Substitute $ k = 2$ back into equation}
        {x}^{2} + 6x + 11 & = {(x+3)}^{2} + 2
  \intertext{Another example:}
       {x}^{2} + 14x + 30 & =
  \intertext{First: halve $b$ to give us $3$, and then put it into the complete
  square form (remembering the k-value):}
                          & = {(x+7)}^{2} + k
  \intertext{Next calculate $k$}
  30 - {7}^{2} = -30 - 49 & = -19
  \intertext{Substitute $k = -19$ back into equation}
       {x}^{2} + 14x + 30 & = {(x+7)}^{2} - 19
  \intertext{Another example:}     
         {x}^{2} - 2x + 7 & =
  \intertext{First: halve $b$ to give us $3$, and then put it into the complete
  square form (remembering the k-value):}
                          & = {(x-1)}^{2} + k
  \intertext{calculate $k$}
   7 - ({-1}^{2}) = 7 - 1 & = 6
  \intertext{Substitute $k = 6$ back into equation}
         {x}^{2} - 2x + 7 & = {(x-1)}^{2} + 6     
\end{align}
From these 3 examples, the pattern should become evident as a 3 stage process:
\begin{enumerate}
  \item Halve $b$ and put into the complete square form
  \item Calculate $c - {h}^{2} = k$
  \item Substitute $k$ back into equation and rewrite in full.
\end{enumerate}
\newpage
\subsection{Non-monic Case, when a != 1}
If given an equation like
\begin{align}
  3{x}^{2} + 12x + 27 & =
  \intertext{we can factor out the coefficient $a$ and the complete the square
  as in a general case}
  3{x}^{2} + 12x + 27 & = 3({x}^{2} + 4x + 9) \\
                      & = 3({(x+2)}^{2} + 5) \\
                      & = 3({(x+2)}^{2}) + 15
  \intertext{this gives rise to the form:}
  a{(x-h)}^{2} + k &
\end{align}
\subsection{Completing The Square Formulae}
\begin{align}
  \intertext{When $a = 1$}
     {x}^{2} + bx + c & = {(x - \frac{-b}{2})}^{2} + k \\
                      & = {(x + \frac{b}{2})}^{2} + (c - \frac{{b}^{2}}{4})
  \intertext{When $ \neq 1$}
    a{x}^{2} + bx + c & = a{(x - \frac{-b}{2})}^{2} + k \\
                      & = a{(x + \frac{b}{2a})}^{2} + (c - \frac{{b}^{2}}{4a})
\end{align}
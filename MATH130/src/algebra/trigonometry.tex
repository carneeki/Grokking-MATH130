%-----------------------------------------------------------------------------%
%- Algebra :: Trigonometry ---------------------------------------------------%
%-----------------------------------------------------------------------------%
\chapter{Trigonometry}
\label{chap:Trigonometry}
\emph{Trigonometry is about ratios of angles and sides}.
Specifically, trigonometry is about the ratios of angles and sides in triangles
with respect to each other when drawn inside a special circle called \emph{the
unit circle}.

A brief interlude on language. Most of our trigonometric measurements are
\emph{NOT} going to be measured in degrees. Most of our measurements of angles
are measured in radians. A \emph{radian} is the ratio of angles with respect
pi.\\
To convert between radians and degrees use the following formulae:

\begin{align}
  \theta \ndeg & = r \times \frac{180\ndeg}{\pi} \\
            r & = \theta \ndeg \times \frac{\pi}{180\ndeg}
\end{align}

A more formal definition of the radian: \emph{The radian measure of an angle
$\theta$ is the arclength subtended by the angle in a unit circle.} This
definition requires we define what a unit circle is. \footnote{This works fine
for \emph{anti-clockwise} angles, however, a radian can never be negative
because a length (the arclength in this case) can never be less than zero.}

A full circle:
\begin{align}
  r & = \theta \ndeg \times \frac{\pi}{180\ndeg} \\
  r & = 360 \ndeg \times \frac{\pi}{180\ndeg} \\
  r & = 2 
\end{align}

\newpage
\section{The Unit Circle}
\label{sec:TheUnitCircle}
The unit circle is such an important part of trigonometry that it is worth
looking it up on Wikipedia \emph{AND} as many other places as possible.

%\begin{figure}[!hbt]
%\label{fig:UnitCircle}
% \begin{tikzpicture}[scale=5.3,cap=round,>=latex]
%   % draw the coordinates
%   \draw[->] (-1.5cm,0cm) -- (1.5cm,0cm) node[right,fill=white] {$x$};
%   \draw[->] (0cm,-1.5cm) -- (0cm,1.5cm) node[above,fill=white] {$y$};
% 
%   % draw the unit circle
%   \draw[thick] (0cm,0cm) circle(1cm);
% 
%   \foreach \x in {0,30,...,360} {
%     % lines from center to point
%     \draw[gray] (0cm,0cm) -- (\x:1cm);
%     % dots at each point
%     \filldraw[black] (\x:1cm) circle(0.4pt);
%     % draw each angle in degrees
%     \draw (\x:0.6cm) node[fill=white] {$\x^\circ$};
%             
%     % 'drop a vertical' as Fran Griffin calls it
%     % x-coord = cos(A)
%     % y-coord = 0;
% %    \draw (\x:0.6cm) -- (\x:0.6cm,0);
%   }
% 
%   % draw each angle in radians
%   \foreach \x/\xtext in {
%     30/\frac{\pi}{6},
%     45/\frac{\pi}{4},
%     60/\frac{\pi}{3},
%     90/\frac{\pi}{2},
%     120/\frac{2\pi}{3},
%     135/\frac{3\pi}{4},
%     150/\frac{5\pi}{6},
%     180/\pi,
%     210/\frac{7\pi}{6},
%     225/\frac{5\pi}{4},
%     240/\frac{4\pi}{3},
%     270/\frac{3\pi}{2},
%     300/\frac{5\pi}{3},
%     315/\frac{7\pi}{4},
%     330/\frac{11\pi}{6},
%     360/2\pi
%   }
%   \draw (\x:0.85cm) node[fill=white] {$\xtext$};
% 
%   \foreach \x/\xtext/\y in {
%     % the coordinates for the first quadrant
%     30/\frac{\sqrt{3}}{2}/\frac{1}{2},
%     45/\frac{\sqrt{2}}{2}/\frac{\sqrt{2}}{2},
%     60/\frac{1}{2}/\frac{\sqrt{3}}{2},
%     % the coordinates for the second quadrant
%     150/-\frac{\sqrt{3}}{2}/\frac{1}{2},
%     135/-\frac{\sqrt{2}}{2}/\frac{\sqrt{2}}{2},
%     120/-\frac{1}{2}/\frac{\sqrt{3}}{2},
%     % the coordinates for the third quadrant
%     210/-\frac{\sqrt{3}}{2}/-\frac{1}{2},
%     225/-\frac{\sqrt{2}}{2}/-\frac{\sqrt{2}}{2},
%     240/-\frac{1}{2}/-\frac{\sqrt{3}}{2},
%     % the coordinates for the fourth quadrant
%     330/\frac{\sqrt{3}}{2}/-\frac{1}{2},
%     315/\frac{\sqrt{2}}{2}/-\frac{\sqrt{2}}{2},
%     300/\frac{1}{2}/-\frac{\sqrt{3}}{2}
%   }
%         
%   \draw (\x:1.25cm) node[fill=white] {$\left(\xtext,\y\right)$};
% 
%   % draw the horizontal and vertical coordinates
%   % the placement is better this way
%   \draw (-1.25cm,0cm) node[above=1pt] {$(-1,0)$}
%     (1.25cm,0cm)  node[above=1pt] {$(1,0)$}
%     (0cm,-1.25cm) node[fill=white] {$(0,-1)$}
%     (0cm,1.25cm)  node[fill=white] {$(0,1)$};
% \end{tikzpicture}
%\caption{The Unit Circle, based on work from Supreme Ayal, TiKZ Examples
%\cite{HTrCS}}
%\end{figure}

A formal definition of the unit circle:
$cos(\theta)$ is the $x$-value of the intersection of the unit circle with the
angle $\theta$.\\
$sin(\theta)$ is the $y$-value of the intersection of the unit circle with the
angle $\theta$\footnote{These words are not quite complete, and would require
the unit circle picture}.

UTUCDTCTV:\footnote{Use the Unit Circle Definition to Calculate The Value}
\begin{align}
  cos(\frac{-2\pi}{3}) & = \ldots
\end{align}
$\pi$ is $\frac{1}{2}$ circle.\\
$\frac{\pi}{3}$ is $\frac{1}{3}$ of $\frac{1}{2}$ a circle.\\
At this point, there has been no trigonometry\ldots \\
% TODO: diagram of above (2pi / 3), drop verticals, & horizontals
The rest of this is high school geometry to determine angles involving
triangles (such as Pythagoras).
% TODO: include pi/3 complement

\subsection{Properties of The Unit Circle}
\label{sec:Properties of The Unit Circle}

\begin{align}
  -1 \leq & sin(\theta) \leq & 1 \\
\end{align}
% TODO: 10 O'Clock unit circle
$sin(\theta)$ is the $y$-value on the unit circle which is bound by -1 and 1.

\begin{align}
  sin(-\theta) & = - sin(\theta) \\
\end{align}

% TODO: Picture of right angle triangle
% TODO: Angle sum of a triangle against 2 parallel lines = pi radians 
% TODO: SOH, CAH, TOA
\subsection{SOH CAH TOA}
\label{sec:SOHCAHTOA}
\begin{align}
  sin(\theta \ndeg) & = \frac{o}{h} \\
  cos(\theta \ndeg) & = \frac{a}{h} \\
  tan(\theta \ndeg) & = \frac{o}{a}
\end{align}

\subsection{Angle Sum Formula}
\label{sec:AngleSumFormula}

There are 3 angle sum formulae to learn, and while the 3rd can be deduced from
the first two, it is easier to just memorise it as the derivation is beyond
the scope of MATH130.

\begin{align}
  sin(\alpha + \beta) & = sin(\alpha)cos(\beta) + cos(\alpha)sin(\beta) \\
  cos(\alpha + \beta) & = cos(\alpha)sin(\beta) - cos(\beta)sin(\alpha) \\
  tan(\alpha + \beta) & = \frac{tan(\alpha) + tan(\beta)}{1 - tan(\alpha)tan(\beta)}
\end{align}

%-----------------------------------------------------------------------------%
%- Algebra :: Simultaneous Equations------------------------------------------%
%-----------------------------------------------------------------------------%
\chapter{Simultaneous Equations}
\label{chap:SimultaneousEquations}

Different cases, but one general strategy. We want to isolate variables, and
substitute them back in to solve for remaining variables.

\begin{enumerate}
  \item One variable, eg $5x = -3$
  \begin{align}
    5x &= -3 \\
    x &= -\frac{3}{5}
  \end{align}
  \item Two variables of linear equations have 3 cases
  \begin{enumerate}
  \item eg Two linear equations in two variables
  \begin{align}
    x - y &= 14 \\
    x + y &= 26 \\
\intertext{Find $x$ and $y$ such that the two equations hold simultaneously}
\intertext{Lines are given by}
    ax +by &= c ~ (\text{for given $a,b,c | a\neq 0, b \neq 0$}) \nonumber \\
\intertext{So a linear equation in 2 variables is the algebraic way of
representing a line in the plane. We want to find the point(s) where these lines
intersect. Rearranging we can get}
    y &= -\frac{a}{b}x + \frac{c}{b} \\
\intertext{where}
    -\frac{a}{b} &= ~ \text{gradient} \\
    \frac{c}{b} &= ~ \text{y-intercept}
  \end{align}
  \item Second case of parallel lines
\begin{align}
  a_1x +b_1 &= c_1 \\
  a_2x +b_2 &= c_2 \\
\intertext{so a point has to satisfy both equations. In equations where there
are equal gradients there is no solution, as the lines never intersect}
  \end{align}
  \item Third case of parallel equal lines
  This case has infinitely many solutions as there really is only one line. 
  \end{enumerate}
\end{enumerate}

\section{Example}
\begin{align}
 x-y &= 14 \\
 x+y &= 26 \\
\intertext{Two operations we are allowed to do. We can multiply an equation by
a non-zero number or we can add /subtract equations.}
  (x-y)+(x+y) &= 14 +26 = 40 \\
  2x &= 40 \\
  x &= 20 \\
  \intertext{substitute this into either equation, by first eq}
  20 -y &= 14 \\
      y &= 6 \\
  \intertext{or into second eq:}
  20 +y &= 26 \\
      y &= 6
\intertext{A unique solution is given by $(x,y)=(20,6)$.}
\end{align}

\subsection{Example}
\begin{align}
  4x -3y &= 14 \\
  9x -4y &= 26 \\
\intertext{Suppose we eliminate $y$ we can cross multiply by the $y$
  coefficients:}
  16x -12y &= 56 \\
  27x -12y &= 78 \\
\intertext{Now we subtract one from the other we eliminate $y$}
  27x-16x -12y --12y &=78 -56 \\
  11x &= 22 \\
  x &= 2\\
\intertext{Having found $x$ we can substitute back into first equation}
  3y &= 4x -16 \\
     &= 8 -16 \\
     &= -6 \\
   y &= -2
\intertext{Or by second equation:}
  4y &= 9x -26 \\
     &= -8 \\ 
   y &= -2 \\
\intertext{Solution given by $(x,y)=(2,-2)$.}
\end{align}
One can verify by substituting both $x,y$ values into either (and both)
original equations to see if they hold true.

\section{Example}
\begin{align}
   4x -3y &= 14 \\
  12x -9y &= 26 \\
  \intertext{Eliminate $y$}
  12x -9y &= 42 \\
  12x -y9 &= 26 \\
  (12x-9y)-(12x-9y) &= 42-26 \\
   12x -9y -12x +9y &= 16 \\
  0 &= 16 \\
\intertext{This does not hold true therefore no solutions.}
\end{align}

\section{Example}
\begin{align}
   4x -3y &= 14 & \label{simeqex_3_eq1}\\
  12x -9y &= 42 & \label{simeqex_3_eq2}\\
\intertext{Eliminate $y$, multiply eq:\ref{simeqex_3_eq1} $\times 3$}
  12x -9y &= 42 & (\ref{simeqex_3_eq1} \times 3)\\
  12x -9y &= 42 & \text{(by eq:\ref{simeqex_3_eq1})} \nonumber
\intertext{Same equation therefore infinite solutions. For each $x$ there is a
corresponding $y$ given by either eq:\ref{simeqex_3_eq1} or
eq:\ref{simeqex_3_eq2}, or a rearrangement like}
  y &= \frac{4}{3}x - \frac{14}{3} &
\end{align}

These relatively simply simultaneous equations can be solved using 3 methods,
elimination (as demonstrated), substitution (TODO) and matrices, in chapter
\ref{chap:Matrices} on page \pageref{chap:Matrices}.

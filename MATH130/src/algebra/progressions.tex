%-----------------------------------------------------------------------------%
%- Algebra :: Arithmetic & Geometric Progressions ----------------------------%
%-----------------------------------------------------------------------------%
\chapter{Progressions}
\label{chap:Progressions}
Progressions are when you have a bunch of numbers in a group, and they are
related to eachother in an order, such as ``add one to the last one'', or
``divide by one plus the last one.'' They fall into two types which resemble the
form:\footnote{again, there are other types, but they fall out of the scope of
MATH130. Some examples include Harmonic Progressions, and Taylor series}

\begin{description}
  \item[Arithmetic Progressions] $a_1 = (1-1)d, a_2 = (2-1)d, a_3 = (3-1)d,
  \ldots$ Essentially, a bunch of numbers separated by a common delta (or
  difference). They sometimes also go by the term \emph{arithmetic sequence}. I
  shall use the term progression however, as it feels less confusing to me when
  we enter the topic of sums.
  \item[Geometric Progressions] $a, ar, ar^2, ar^3, ar^4, \ldots$, that is, a
  quotient of consecutive numbers is constant, we divide $\frac{ar^2}{ar} = a$,
  and $\frac{ar^3}{ar^2} = a$, and $\frac{ar^4}{ar^3} = a$.
\end{description}
If you cannot recognise a form of numbers, you can perform what I call the
``double difference test''. We can taking any $2$ consecutive numbers, and
compare their difference with any other $2$ consecutive numbers.
\begin{enumerate}
  \item If the delta is identical between both pairs, you have an arithmetic
  progression.
  \item If the delta is different between both pairs, you have a geometric
  progression.
\end{enumerate}

\section{Arithmetic Progressions}
\label{ArithmeticProgressions}
Arithmetic progressions are really nothing more than a set of numbers which have
a difference from one number to the next of a common delta, ($d$), as seen in
the double difference test. An arithmetic progression takes the form:
\begin{align}
  a_n &= a_1 + (n-1)d
\intertext{or more generally:}
  a_n &= a_m + (n-m)d \label{eq:ArithmeticProgression}
\intertext{Sometimes knowing the in the form of $n=$ is handy, this is derived
in the subsection the sum of arithmetic progressions:}
  n &= \frac{1}{d}(a_n - a_m) +m
\end{align}
\begin{table}[!htb]
\begin{tabularx}{\linewidth}{| l X |}
\hline
\multicolumn{2}{|l|}{Where:} \\
\hline \hline
$a_n$ & is an arbitrary term we wish to determine \\
$n$   & is the $n$\tsup{th} arbitrary value to index the term \\
$m$   & is an arbitrary term we are given \\ 
$a$   & is the number itself \\
$d$   & is the delta between numbers \\
\hline
\end{tabularx}
\caption{Components to an arithmetic progression.}
\end{table} 
\subsection{Determining the n-th term of an arithmetic progression}
\label{sec:nthTermOfAnArithmeticProgression}
Given some sequence of numbers, $1, 3, 5, 7, 9, \ldots$, we can determine the
delta, $d=2$ (as the difference of any pair of numbers is $2$), and $a_1 = 1$.
Armed with this knowledge we can determine any term. For example the
$1000$\tsup{th} term can be obtained as follows:
\begin{align}
  a_n &= a_m + (n-m)d \nonumber \\
  a_{1000} &= a_{1} + (1000-1)*2 \\
    &= 1 + 999*2 \\
    &= 1 + 1998 \\
    &= 1999
\end{align}
It also works backwards for numbers before first term, suppose given the same
sequence as above we wanted to determine what the $-1000$\tsup{th}
term\footnote{putting my poor language aside, let's pretend that a preceeding
term could take place before the first term.},\tsup{,}\footnote{this is partly
why math is better than English, we can describe things and even poor grammar
can convey the meaning better in math than poor grammar in English.} is.
\begin{align}
  a_n &= a_m + (n-m)d \nonumber \\
  a_{-1000} &= a_{1} + (-1000+1)*2 \\
    &= 1 + (-999)*2 \\
    &= 1 -1998 \\
    &= -1997 \\
\end{align}
Now you might be saying ``whoa hold on a sec, we had $1999$ before, shouldn't we
expect $-1999$?''. The answer is no. The reason: we must traverse across
``zero'' to get to $-1$ as the term before $1$, so our negative number is always
going to be $d$ less than than the positive number (and then multiplied by
$-1$).
\newpage
\subsection{Sum of an Arithmetic Progression - aka Arithmetic Series}
\label{sec:ArithmeticSeries}
For some reason\footnote{no pun intended}, we call a sum of subset of elements
in an arithmetic progression an \emph{arithmetic series}. To determine the sum:
\begin{align}
%   S_n
%     &= a_1 + (a_1 + d) + (a_1 + 2d) + \ldots + (a_1 + (n-2)d) + (a_1 + (n-1)d) \\
%     &= (a_n - (n-1)d) + (a_n - (n-2)d) + \ldots + (a_n - 2d) + (a_n -d) + a_n \\
%   2S_n
%     &= n(a_1 + a_n) \\
  S_n
    &= \frac{n}{2}(a_1 + a_n) \label{eq:ArithmeticSum1}\\
  \intertext{by:}
    & = a + (a+d) + (a+2d) + \ldots + (a + (n-1)d) \\
    &= n\cdot a + (d + 2d + \ldots + (n-1)d) \\
    &= n\cdot a + d(1 + 2 + \ldots + (n-1)) \\
\intertext{if $n-1$ is even:}
    1 + 2 + 3 + 4 & \ldots + (n-4) + (n-3) + (n-2) + (n-1) \nonumber \\
    \frac{n-1}{2} \cdot n &= \frac{n(n-1)}{2} \\
\intertext{if $n-1$ is odd:}
    \frac{n-2}{2} \cdot n &= \frac{n}{2} \\
    \frac{n-2}{2} \cdot \frac{n}{2} &= \frac{1}{2}\cdot \left[(n-2)n + n\right] \\
    &= \frac{1}{2}(n^2 - 2n + n) \\
    &= \frac{1}{2}(n^2 -n) \\
    &= \frac{n(n-1)}{2} \\
    &= \frac{n}{2}(2a_1 + (n-1)d) \label{eq:ArithmeticSum2}
\end{align}

\begin{table}[!htb]
\begin{tabularx}{\linewidth}{| l X |}
\hline
\multicolumn{2}{|l|}{Where:} \\
\hline \hline
$Sn$  & is the sum of an arbitrary term \\
$a_n$ & is the arbitrary term \\
$a_1$ & is the first term \\
$d$   & is the delta between numbers \\
\hline
\end{tabularx}
\caption{Components to an arithmetic series.}
\end{table}
\noindent
Let's take our odd number progression from before: $1, 3, 5, 7, 9, \ldots, 1999$
and determine what the sum of these numbers are.
\begin{align}
\intertext{First we need to find what the value of $n$ in the progression is,
pretending we don't know it from before). We can rearrange
\eqref{eq:ArithmeticProgression} to make $n$ the subject of the equation:}
  a_n
    &= a_m + (n-m)d \nonumber \\
  a_n
    &= a_1 + (n-1)d \nonumber \\
  a_1 + (n-1)d
    &= a_n                       &\text{swap sides}\\
  (n-1)d
    &= a_n - a_1                 &\text{subtract $a_1$}\\
  n-1
    &= \frac{1}{d}(a_n - a_1)    &\text{divide by $d$}\\ 
  n &= \frac{1}{d}(a_n - a_1) +m &\text{add $m$}
  \intertext{Substitute values:}
  n &= \frac{1}{2}(1999 - 1) + 1 \\
    &= \frac{1}{2}(1998) + 1 \\
    &= 999 + 1 \\
    &= 1000
\intertext{Now we can substitute into the sum equation
\eqref{eq:ArithmeticSum1}}
  S_n
    &= \frac{n}{2}(a_1 + a_n) \nonumber \\
  S_{1000}
    &= \frac{1000}{2}(1 + 1999) \\
    &= 500 \cdot 2000 \\
    &= 1 000 000
\end{align}
\newpage
\section{Geometric Progressions}
\label{GeometricProgressions}
Geometric progressions are really nothing more than a set of numbers which have
a ratio-based difference from one number to the next of a common ration, ($r$).
If you apply the double difference test, a geometric progression will fail,
however if you either divide two pairs of numbers by their previous or next
consecutive number (a ``double quotient test''), you should have the same ratio.

A finite geometric progression of numbers will look like $ar^0, ar^1, ar^2,
\ldots ar^{n-1}$
\begin{align}
  a_n &= ar^{n-1} \label{eq:GeometricProgression}
\intertext{Sometimes knowing the in the form of $n=$ is handy, we can do this
with logarithms (see section \ref{sec:LogLaws} for a list of log laws - many
of which should be committed to memory)}
  \frac{a_n}{a} &= r^{n-1} \\
  \log_r(\frac{a_n}{a}) & = \log_r(n-1) \\
  \log_r(a_n) - \log_r(a) & = \log_r(n-1) \\
  a_n - a &= n -1 \\
  a_n - a + 1 & = n \\
\intertext{From this form, getting $a$ is a matter of rearranging:}
  a_n - n & = a
\end{align}
\begin{table}[!htb]
\begin{tabularx}{\linewidth}{| l X |}
\hline
\multicolumn{2}{|l|}{Where:} \\
\hline \hline
$a$ & is the first term of the geometric progression \\
$a_n$ & is an arbitrary term we wish to determine \\
$r$ & is the ratio between numbers \\ 
$n$ & is the $n$\tsup{th} arbitrary value to index the term \\
\hline
\end{tabularx}
\caption{Components to a Geometric Progression.}
\end{table} 
\subsection{Determining the n-th term of a geometric progression}
\label{sec:nthTermOfAnGeometricProgression}
Given some sequence of numbers, $1, 2, 4, 8, 16, 32, \ldots$, we can see the
formula follows: $1=ar^{(1-1)}=ar^0=a*1=a=1$, and we can see that the
ratio is that of doubling ($r=2$), such that $1=a2^{(1-1)}=a2^0=a*1=a=1$. Armed
with this knowledge we can determine any term. For example the $10$\tsup{th}
term can be obtained as follows:
\begin{align}
  a_n &= ar^{n-1} && \text{by \eqref{eq:GeometricProgression}} \nonumber \\
  a_{10} &= 1*2^{10-1} \\
    &= 1*2^{10-1} \\
    &= 1*2^9 \\
    &= 512
\end{align}
Another example that is easily worked out using finders and no calculator.
\begin{align}
    a_n &= ar^{n-1} && \text{by \eqref{eq:GeometricProgression}} \nonumber \\
  a_{5} &= 1*2^{5-1} \\
    &= 1*2^{5-1} \\
    &= 1*2^4 \\
    &= 16
\end{align}
Simply raise one finger\footnote{ideally your thumb or pinky!}, and each time
you double, raise another finger. If you have the normal number of digits per
hand, the $5$\tsup{th} will match the result.
\newpage
\subsection{Sum of a Geometric Progression - aka Geometric Series}
\label{sec:GeometricSeries}
We call a sum of subset of elements in an geometric progression a
\emph{geometric series}. To determine the sum:
\begin{align}
  S_n
    &= ar^0 + ar^1 + ar^2 + \ldots + ar^{n-1} \label{eq:GeometricSum1}
\intertext{If we multiply both sides by $r$}
  rS_n
    &= ar^1 + ar^2 + \ldots + ar^{n-1} + ar^n \\
\intertext{and we take the difference of these two equations:}
  S_n - rS_n
    &= ar^0 - ar^n \\
    &= a - ar^n
\intertext{$S_n$ can also be expressed as:}
  S_n 
    &= \frac{a-ar^n}{1-r} \\
\intertext{Because (and it's easier to understand why when you see this
working):}
  S_n
    &= \frac{S_n - rS_n}{x} \\
  \text{Let $S_n=y$} \nonumber \\
   y &= \frac{y-(ry)}{x} \\
  yx &= y-yr \\
   x &= 1-r
\intertext{Note: it is important that $r \neq 1$, else $S_n$ is undefined as
such the progression looks like:}
  & a, a, a, a, \ldots, a \\
\intertext{So the sum looks like}
  S_n &= n \cdot a
\end{align}

\begin{table}[!htb]
\begin{tabularx}{\linewidth}{| l X |}
\hline
\multicolumn{2}{|l|}{Where:} \\
\hline \hline
$Sn$ & is the sum of an arbitrary term \\
$n$  & is the $n$\tsup{th} index of the term we are interested in \\
$a$  & is the first term \\
$r$  & is the ratio between numbers \\
\hline
\end{tabularx}
\caption{Components to a geometric series.}
\end{table}
%-----------------------------------------------------------------------------%
%- Algebra :: Types of Numbers & Symbols Used --------------------------------%
%-----------------------------------------------------------------------------%
\chapter{Types of Numbers \& Symbols Used}
\label{chap:TypesOfNumbersAndSymbolsUsed}
Maths is pretty much a written language used to convey what people want to do
to numbers, variables, or other bits of information. There are various types
of numbers, and sometimes special symbols are used to denote what conditions
must be placed on those numbers.
\section{Types of Numbers}
\label{sec:TypesOfNumbersUsed}
Table \ref{tab:TypesOfNumbers} outlines the types of numbers encountered in
MATH130, followed by a few additional types of numbers that are handy to bear
in mind.
\begin{table}[!htb]
\label{tab:TypesOfNumbers}
\begin{tabularx}{\linewidth}{| c | l | c | X |} \hline
  Symbol & Name & In MATH130 & Description \& Example \\ \hline \hline
  $\mathbb{N}$ & Natural Number     & Yes & Any whole number greater than
                                            zero. $1, 2, 3 $ \\ \hline
  $\mathbb{R}$ & Real Numbers       & Yes & Any number along a continuum.
                                            $-1, 0, 1, \pi $ \\ \hline
  $\mathbb{Z}$ & Integers           & Yes & Any whole number.
                                            $-1, 0, 1 $ \\ \hline
  $\mathbb{I}$ & Irrational Numbers & Yes & Numbers which cannot be expressed
                                            as a fraction.
                                            $e, \pi, \sqrt[2]{2} $ \\ \hline
  $\mathbb{Q}$ & Rational Numbers   & Yes & Numbers which can be represented as a
                                            fraction.
                                            $ \frac{1}{2}, 1, \frac{0}{4} $
                                            \\ \hline
  $\mathbb{C}$ & Complex Numbers    & Yes \footnote{Only a brief introduction
  to complex numbers is covered, and traditionally it is at the end of MATH130
  incase there isn't enough time to cover it.}
                                            & Numbers which have both a real
                                            part, and an imaginary part.
                                            $ \frac{2}{-1} = i $ \\ \hline
  $\mathbb{P}$ & Prime Numbers      & No \footnote{They come up, but really, it's
 only in factorization.}
                                            & Numbers which are divisible only
                                            by themselves and one.
                                            $1, 2, 3, 5, 7, 11, 13 $ \\  \hline
\end{tabularx}
\caption{Types Of Numbers}
\end{table}

\section{Types of Symbols}
\label{sec:TypesOfSymbols}
Table \ref{tab:TypesOfSymbols} outlines the types of symbols you are likely
to encounter in MATH130. This list is partially built from \citet{IjPb7}.
\begin{table}[!htb]
\label{tab:TypesOfSymbols}
\begin{tabularx}{\linewidth}{| c || c | X |}
  \hline
  Symbol & Example & Read as \\ \hline \hline
  $ \in    $ & $ x \in    \mathbb{R} $ & x is an element of $\mathbb{R}$ (real
                                       numbers)                       \\ \hline
  $ \notin $ & $ x \notin \mathbb{R} $ & x is not an element of $\mathbb{R}$
                                       (real numbers)                 \\ \hline
  $ \cup   $ & $ \mathbb{I} \cup \mathbb{Q} \in \mathbb{R}$
                                       & The set $\mathbb{I}$ (irrational
                                       numbers) and $ \mathbb{Q} $ (rational
                                       numbers) are in set $\mathbb{R}$ (real
                                       numbers).                      \\ \hline
  $ =       $ & $ x = y       $ & x is equal to y                     \\ \hline
  $ \neq    $ & $ x \neq    y $ & x is not equal to y                 \\ \hline
  $ \approx $ & $ x \approx y $ & x is approximately equal to y       \\ \hline
  $ \equiv  $ & $ x \equiv  y $ & x is equivalent to y                \\ \hline
  $ <       $ & $ x < y       $ & x is less than y                    \\ \hline
  $ >       $ & $ x > y       $ & x is greater than y                 \\ \hline
  $ \leq    $ & $ x \leqslant $ & x is equal to or less than y        \\ \hline
  $ \geq    $ & $ x \geqslant $ & x is equal to or greater than y     \\ \hline
  $ f(x)    $ & $ f(x) = mx + b $ & function of x is equal to mx + b  \\ \hline
  $ \sum    $ & $ \sum_{i=1}^{10} t_i $ & sum of terms t for values 1 to 10
                                                                      \\ \hline
  $ \int    $ & $ \int_0^\infty e^{-x}\,\mathrm{d}x $
                                & integrate $ e^{-x} $ from 0 to $ \infty $
                                with respect to x                     \\ \hline
  $ \derivative f(x) $ & $ \derivative f(x) $ & differentiate $f(x)$ with
                                                respect to x.                     
                                                \\ \hline
  $ y'~\mathrm{d}x $ & $ y'~\mathrm{d}x   $ & differentiate $y$ with respect to
                                            x.   
                                            \\ \hline
\end{tabularx}
\caption{Types of mathematical symbols}
\end{table}

\section{A Brief Interlude on Language}
\label{sec:ABriefInterludeOnLanguage}
An expression and an equation are \emph{different} things, despite the fact that
they look similar. An expression might be something like $3x - 8$, however, $x$
has no value. By the fact that $3x - 8$ has no value \emph{assigned} to it, it
is an expression. If we were to assign a value to it, $3x-8 = 0$ then we can
call it an equation.

\begin{align}
  3x - 8 & & \text{an expression} \nonumber   \\
  3x - 8 & = 0 & \text{an equation} \nonumber \\
\end{align}
%-----------------------------------------------------------------------------%
%- Algebra :: Arithmetic & Geometric Progressions ----------------------------%
%-----------------------------------------------------------------------------%
\chapter{Counting Techniques}
\label{chap:CountingTechniques}
How do we count certain events?

Notation:
\begin{align}
  n! & = n(n-1)(n-2)\ldots 3 \cdot 2 \cdot 1 \left\{
    \begin{array}{c c}
      n \in \mathbb{N} \\
      \mathbb{N} \in \mathbb{Z} & | ~ n > 0 \\
    \end{array}
  \right.
\end{align}

Example:
\begin{align}
  5! &= 5 \cdot 4 \cdot 3 \cdot 2 \cdot 1 \\
     &= 120
\end{align}

Example: The alphabet has 26 letters. How many ways can I choose 1 letter in the
English alphabet?

1 letter:
\begin{align}
  26 \cdot 1 &= 26
\end{align}

How many ways can I form words of 2 letter in the English alphabet?

2 letters:
\begin{align}
  26 * 25 & = \ldots
\end{align}

How many ways can I form choose 2 distinct letters in the English alphabet?
1 letter:
\begin{align}
  \frac{26\cdot25}{2} + &= \ldots
\end{align}

Example: Group of 20 people. Choose 4 people and seat them in a row:

\begin{align}
  (20 \cdot 19 \cdot 18 \cdot 17) &= \ldots \text{distinct arrangements}
\end{align}

Now suppose they are seated around a square:
\begin{align}
  \frac{(20 \cdot 19 \cdot 18 \cdot 17)}{4} &= \ldots
\end{align}

In how many ways can we chose 4 people in a group of 20 where order is not
important?
\begin{align}
  (20 \cdot 19 \cdot 18 \cdot 17) - (4 \cdot 3 \cdot 2 \cdot 1) &= \ldots
    &= \frac{(20 \cdot 19 \cdot 18 \cdot 17)}{4!} \\
    &= \frac{(20 \cdot 19 \cdot 18 \cdot 17)}{4!} \cdot \frac{16!}{16!} \\
    &= \frac{20!}{4! \cdot 16!} \\
    &= C(20,4) \\
    &= C(20,16)
\end{align}

We start with all distinct possibilities and then remove the number of distinct
arrangements of the select 4. $C(h,n)$ means \emph{``Choose $n$ needles from
the haystack, $h$''}. This is called a \emph{binomial coefficient}\footnote{and
is covered in more detail in DMTH137 and MATH135}.

Choosing $r$ objects from a collection of $n$ objects:
\begin{enumerate}
  \item Without replacement
  \begin{enumerate}
    \item Order important: $n \cdot (n-1) \cdot (n-2)\cdot \ldots \cdot (n-r+1) $
    \item Order not important: $\frac{n \cdot (n-1) \cdot (n-2)\cdot \ldots \cdot (n-r+1)}{r!}$
  \end{enumerate}
  \item With replacement
  \begin{enumerate}
    \item Order important: $n \cdot n \cdot n \cdot \ldots \cdot n_r = n^r $
    \item Order not important: $C(n+r-1, r) = \frac{n+r-1}{r!(n-1)!}$ (mainly in DMTH137)
  \end{enumerate}
\end{enumerate}

Permutations:
\begin{align}
  n(n-1)(n-2)\ldots(n-r+1)\cdot\frac{(n-r)!}{(n-r)!} = \frac{n!}{(n-r)!} \\
\intertext{which is the same as without replacement, order important and is 
represented by the symbol}
 P(n,r) &= \frac{n!}{(n-r)!} \\
 (_r^n) &= \frac{n!}{r!(n-r)!} \\
 &= \frac{n!}{(n-r)!(n-(n-r))!} \\
 &= (_{n-r}^n) \\
 &= C(n,n-r)
\end{align}
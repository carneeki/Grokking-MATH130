%-----------------------------------------------------------------------------%
%- Algebra :: Inequalities ---------------------------------------------------%
%-----------------------------------------------------------------------------%
\chapter{Inequalities}
\label{chap:Inequalities}
If $a \leq b$, then $a+c \leq b+c$ \\
If $a < b$, then $a+c < b+c$ \\
\\
If $a \leq b$ and $c > 0$, then $a \cdot c \leq b \cdot c$ \\
If $a < b$ and $c > 0$, then $a \cdot c < b \cdot c$ \\

However, when we multiply or divide by a negative, change the sign of the
inequality.
\section{Example}
\begin{align}
  -3x & > 9 \\
\intertext{$\div$ both sides by  $-3$}
  x & > -3~ \text{is NOT correct} \nonumber \\
  x & < -3 ~ \text{is correct}
\end{align}
Consider
\begin{align}
  \frac{3}{x-1} & > 4 \\
  3 & > 4(x-1) \\
  3 & > 4x - 4 \\
  3 + 4 & > 4x \\
  7 & > 4x \\
  \frac{7}{4} & > x \\
  \therefore x & < \frac{7}{4}
\end{align}
However, this presents a problem; if we substitute $x=1$ in, we have a
``disaster'', with a quotient of $0$.
\begin{align}
  \frac{3}{1 - 1} & > 4~\text{cannot possibly be a solution}
\end{align}
Try $x=0$
\begin{align}
  \frac{3}{0 - 1} & > 4 \\
  \frac{3}{-1} & > 4 \\
  -3 & > 4~\text{which we know to be false}
\end{align}
Let's ask ourselves ``Why didn't this work?''. What went wrong was that we don't
know whether $x-1$ is positive or negative. There are several resolutions to the
problem.
\begin{enumerate}
  \item Take cases: consider
  \begin{enumerate}
    \item $x-1 > 0$
    \item $x-1 < 0$
  \end{enumerate}
  And prove whether they are logically true.
  \item Multiply by a non-negative. In this case $(x-1)^2$ because a square can
  never be a negative number.
  \item Rearrange to a form such as $\frac{A}{B} < 0$ and consider
  signs.\footnote{it could also be $\frac{A}{B} > 0$}
  \item Plot the inequality: $y = \frac{3}{x-1}$ and see where the $y$ value is
  bigger than $4$.
\end{enumerate}

First way with cases:
\begin{align}
  \frac{3}{x-1} & > 4 \\
  \text{Assume } x-1 & > 0 \\
  3 & > 4(x-1) \\
    & > 4x - 4 \\
  7 & > 4x \\
  \frac{7}{4} & > x \\
  x & < \frac{7}{4} \\
  \text{We only keep x's where $x-1 > 0$, $x > 1$} \nonumber \\
  \text{Assume } x-1 & < 0 \\
  \frac{3}{x-1} & < 4 \\
  x & > \frac{7}{4} \\
  \text{We only keep x's where $x-1 < 0$, $x < 1$} \nonumber
\end{align}
So we combine our answer is when we take the answers that are logically
acceptable.

Second way with squares:
\begin{align}
  \frac{3}{x-1} & > 4 \\
  (x-1)^2 \times \frac{3}{x-1} & > 4 (x-1)^2 \\
  (x-1)3 & > 4(x-1)^2 \\
  0 & > 4(x-1)^2 - 3(x-1) \\
    & > (x-1)(4(x-1) - 3) \\
    & > (x-1)(4x - 7) \\
\text{we can plot this} \\
 \therefore 1 < x < \frac{7}{4} \\
\end{align}
% INCLUDE THE PLOT
3rd Way: (was skipped for tutorial exercise)

Graphical Way:
Sketch $y = \frac{3}{x-1}$ \\
Vertical asymptote $x = 1$.
\begin{align}
  x = 1 + \text{a small number} \\
  y = \text{BIG}
\end{align}
% INCLUDE THE PLOT


A second problem:
\begin{align}
  \frac{4x}{2x+3} > 2 \\
\end{align}

By squares (quadratic) method:
\begin{align}
  \frac{4x}{2x+3} > 2 \\
  (2x+3)^2 \times \frac{4x}{2x+3} & > 2(2x+3)^2 \\
  4x \times (2x+3) & >  2(2x+3)^2 \\
  0 & > 2(2x+3)^2 - 4x(2x+3) \\
    & > (2x+3)(2(2x+3) -4x) \\
    & > (2x+3)(4x +6 -4x) \\
    & > (2x+3)(6) \\
    & > 12x + 18 \\
    -12x & > 18 \\
    12x  & < -18 \\
       x & < -\frac{3}{2}
    \therefore x < -\frac{3}{2}
\end{align}

By signs:
\begin{align}
  \frac{4x}{2x+3} -2 > 0 \\
  \frac{4x}{2x+3} -\frac{2x+3}{2x+3} > 0 \\
  \frac{-6}{2x+3} & > 0 \\
  -6 \text{always -ve} \\
  2x + 3 & < 0 \\
  2x &< -3 \\
  x & < \frac{-3}{2} \\
\end{align}

\section{Cauchy-Schwartz Inequalities}
\begin{align}
  (a-b)^2 &\geq 0 \\
  a^2 + b^2 -2ab &\geq 0 \\
  a^2 +b^2 &\geq 2ab %~\text{\textbb{ALWAYS}}
\end{align}

\subsection{Examples}
Solve $2 -x -x^2 \geq 0$
\begin{align}
2 -x -x^2 \geq 0 \\
\intertext{we can factorize}
  2 -x -x^2 &= -(x+2)(x-1) \geq 0 \\
  &= (x+2)(1-x) \geq 0 \\
\intertext{so either}
  x+2 &\geq 0 ~ \text{and} ~ 1-x \geq 0 \\
\intertext{or}
  x+2 &\leq 0 ~ \text{and} ~ 1-x \leq 0 \\
\intertext{first case:}
  x & \geq -2 ~\text{and}~ 1 \geq x \\
  -2 & \leq x \leq 1 \\
\intertext{second case:}
  x &\leq -2 ~\text{and}~ x \geq 1 ~\text{, no additional solutions}
\end{align}
so, $2-x-x^2$ iff $-2 \leq x \leq 1$.

\subsection{Example - TODO: finish from frank's notes}
Solve 
\begin{align}
  2 &\leq& \frac{4x-2}{x+4} &<& 3 \\
\intertext{case where x+4 > 0}
x + 4 &>& 0 2(x+4) \leq 4x-2 < 3(x+4) \\
\intertext{case where x+4 < 0}
x+4 < 0 : 2(x+4) \geq 4x-2 > 3(x+4)
\end{align}
% preamble.tex v 2011-10-31 2216AEDT
%- (02) Document Class and Options -------------------------------------------%
\documentclass[
%  pagesize,
  a4paper,
  pdftex,
%  fontsize=11pt,
%  draft=true,
  twoside,
]{book}
%
%\clip(-11.74,-31.43) rectangle (11.68,35.96);
%- (03) Standard package includes and options --------------------------------%
%\usepackage{draftwatermark} % draft watermark. Comment these 2 lines in final
%\SetWatermarkLightness{0.9} %
\usepackage{watermark}
\usepackage{amsmath}       % amsmath & amssymb are almost ALWAYS required.
\usepackage{amssymb}       %
\usepackage{gensymb}       % for the degrees circle!
%
%\usepackage{verbatim}     % multiline commenting ( c++ equiv /* ... */ )
%
\usepackage{color}
\usepackage{xcolor}         % pdflatex
\definecolor{neekiGrey}{RGB}{192,192,192}
\definecolor{neekiRed}{RGB}{172,40,41}
\definecolor{neekiBlue}{RGB}{62,70,157}
\definecolor{neekiGreen}{RGB}{39,171,39}
\definecolor{neekiPurple}{RGB}{105,39,171}
%
\usepackage{geometry}      % option for altering page dimensions if needed
%\usepackage[pdftex]{graphicx} % including image files for figures (ie
                              % non-[E]PS)
%                          % valid types: jpeg, png, pdf
\usepackage{wrapfig}       % the figures themselves
\usepackage[numbers,
square,
longnamesfirst
]{natbib}                  % prettybib
\usepackage[pdftex]{hyperref} % clickable TOC and refs
%\usepackage[all]{xy}      % category theory helpers
%\xyoption{all}            % category theory helpers
%\input xy                 % category theory helpers
\usepackage{pgf,tikz}      % easy graphic thing
\usepackage{tabularx}    % easy tables
\usepackage{url}         % easy urls
\usepackage{multirow}    %
\usepackage{lipsum}      % autogen placeholder text
\usepackage{datetime}
\usepackage{listings}
%
%- (04) Custom Definitions and Alterations -----------------------------------%
\usepackage[T1]{fontenc} % font-doohickey
%\usepackage{tgadventor}  % font
%\usepackage[math]{iwona} % font
\usepackage{kurier} % font

\usetikzlibrary{arrows}

%
\linespread{1.5} % carneeki@ use approx 150% line spacing just like MaxDesign.
\hypersetup{
  % DO NOT CHANGE THESE
%   pdftitle={\metaTitle},
%   pdfauthor={\metaAuthorShort},
%   pdfsubject={\metaSubject},
%   pdfkeywords={\metaKeywords},
  pdfcreator={LaTeX},
  pdfproducer={LaTeX},
  pdftoolbar=false,
  % But change these to taste:
  pdffitwindow=false,   % window fit to page when opened
  pdfstartview={Fit},   % fits the width of the page to the window
                        % all (useful) opts: Fit, FitH, FitV,
  pdfpagelayout=TwoPageLeft,
  pdfnewwindow=true,    % open links in new window
  colorlinks=true,      % false = boxed links; true = colored links
  linkcolor=neekiRed, % color of internal links
  citecolor=neekiBlue, % color of links to bibliography
  filecolor=red, % color of file links
  urlcolor=red  % color of external links
}
\usepackage[Lenny]{fncychap}
%\usepackage[Bjornstrup]{fncychap}
%
%- (05) Custom Commands ------------------------------------------------------%
% MATH
%\newcommand{\derivative}[1][x]{\frac{\mathrm{d}}{\mathrm{d}#1}}
\newcommand{\md}[1]{\mathrm{d}#1}
\newcommand{\deriv}[2]{\frac{\md{#1}}{\md{#2}}} % derivative of fn
\newcommand{\lderiv}[2]{\deriv{}{#2}(#1)} % deriv of a long fn
\newcommand{\myint}[4]{\int_{#1}^{#2} #3~\md{#4}} % integrate
\newcommand{\aderiv}[2]{\myint{}{}{#1}{#2}}
\newcommand{\ndeg}{^\circ} % the degrees symbol
\newcommand{\neqref}[1]{\text{(#1)}}
% trig
\newcommand{\rsin}{\sin^{-1}}
\newcommand{\rcos}{\cos^{-1}}
\newcommand{\rtan}{\tan^{-1}}
\newcommand{\arcsec}{\operatorname{arcsec}}
\newcommand{\arccsc}{\operatorname{arccsc}}
\newcommand{\arccot}{\operatorname{arccot}}
% DMTH
\newcommand{\lxor}{\oplus}
\newcommand{\lxnor}{\lnot\lxor}
\newcommand{\true}{(\mathbb{T})}
\newcommand{\false}{(\mathbb{F})}
% GENERAL
\newcommand{\tsup}[1]{\textsuperscript{#1}}
\newcommand{\tsub}[1]{\textsubscript{#1}}
\newcommand{\qedbitches}{\qed}
\renewcommand{\labelenumi}{(\alph{enumi})}
\renewcommand{\labelenumii}{(\roman{enumii})}
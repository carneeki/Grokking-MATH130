%-----------------------------------------------------------------------------%
%- Calculus :: Differentiation of Trigonometric Functions --------------------%
%-----------------------------------------------------------------------------%
\chapter{Differentiation of Trigonometric Functions}
\label{chap:DifferentiationOfTrigFunctions}
Consider the plot of the function $f(x) = \sin(x)$
\begin{figure}[!htb]
\label{fig:GraphTemplate}
\begin{tikzpicture}[samples=50,scale=1]
   \draw[color=neekiRed,domain=0:pi,variable=x]
     plot[id=sinx] function{1/x}
     node[right] {$f(x) = \frac{1}{x}, x \in \{ -5,-\frac{1}{5} \} $};
%   \draw[color=neekiBlue,domain=0:5,variable=x]
  \draw[<->]  (0.0,-5.0) -- (0.0,5.0) node[above] {$f(x)$};
  \draw[<->] (-5.0, 0.0) -- (5.0,0.0) node[right] {$x$};
\end{tikzpicture}
\caption{A hyperbolic function: $f(x) = \frac{1}{x}$}
\end{figure}
A hyperbolic function in the form $ f(x) = \frac{1}{x}$
\begin{table}[!hbt]
\label{tab:GraphTemplateParts}
\begin{tabularx}{\linewidth}{| l X |}
  \hline
  \multicolumn{2}{|l|}{Where:} \\
  \hline \hline
  x & is the independent variable\\
  y & is the dependent variable\\
\hline
\end{tabularx}
\end{table}

Here are some equations that are handy to remember. Source \cite{RHBDiffQuickStart}.
\begin{align}
  \deriv{x^n}{x}                & = n{x}^{n-1} \\
  \deriv{{\emph{e}}^{ax}}{x} & = a{\emph{e}}^{ax} \\
  \deriv{\ln(x)}{x}          & = \frac{1}{x} \\
  \deriv{\sin(ax)}{x}        & = (a)\cos(ax) \\
  \deriv{\cos(ax)}{x}        & = (-a)\sin(ax) \\
  \deriv{\tan(ax)}{x}        & = (a)\sec^2(ax) \\
  \deriv{\sec(ax)}{x}        & = (a)\sec(ax)\tan(ax) \\
  \deriv{\csc(ax)}{x}        & = (-a)\csc(ax)\cot(ax) \\
  \deriv{\cot(ax)}{x}        & = (-a)\csc^2(ax) \\
  \deriv{\rsin(\frac{x}{a})}{x}
    & = \frac{1}{\sqrt[2]{{a}^{2} - {x}^{2}}} \\
  \deriv{\rsin(\frac{x}{a})}{x}
    & = \frac{-1}{\sqrt[2]{{a}^{2} - {x}^{2}}} \\
  \deriv{\rsin(\frac{x}{a})}{x}
    & = \frac{a}{\sqrt[2]{{a}^{2} - {x}^{2}}}
\end{align}
For a proof, refer to appendix section \ref{sec:DiffTrigProof},
``Differentiation of Trig Functions Proof''
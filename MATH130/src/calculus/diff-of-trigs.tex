%-----------------------------------------------------------------------------%
%- Calculus :: Differentiation of Trigonometric Functions --------------------%
%-----------------------------------------------------------------------------%
\chapter{Differentiation of Trigonometric Functions}
\label{chap:DifferentiationOfTrigFunctions}
This section combines aspects of the unit circle with aspects of
differentiation. A solid foundation in both of these may not be necessary, but
it will certainly help. Consider the plot of the function $f(x) = \sin(x)$.
While doing so, recall what the derivative actually is\footnote{It is the rate
of change of something, and it is represented by the slope of the graph you are
looking at. If this is at all foreign, consider looking at Chapter
\ref{chap:Differentiation}.}. In particular, pay attention to \emph{how the
slope changes as you move back and forth along the graph}. 
\begin{figure}[!htb]
\label{fig:GraphTemplate}
\begin{tikzpicture}[samples=1000,domain=-6.28:6.28]
  \draw [color=neekiGrey,dash pattern=on 1pt off 1pt,xstep=1cm,ystep=1cm]
  (-7,-3) grid (7,3);
  \draw[->,color=black] (-7,0) -- (7,0);

  % x axis labels are not easily looped sadly
  \draw (-6.283185307,0) node[below]{${-2\pi}$};
  \draw (-4.71238898,0)  node[below]{$-\frac{3\pi}{2}$};
  \draw (-3.14159,0)     node[below]{${-\pi}$};
  \draw (-1.570795,0)    node[below]{$-\frac{\pi}{2}$};
  
  \draw (1.570795,0)     node[below]{$\frac{\pi}{2}$};
  \draw (3.14159,0)      node[below]{${\pi}$};
  \draw (4.71238898,0)   node[below]{$\frac{3\pi}{2}$};
  \draw (6.283185307,0)  node[below]{${2\pi}$};
  \foreach \y in {-1,1}
    \draw[shift={(0,\y)},color=black] (2pt,0pt) -- (-2pt,0pt) node[left]
    {\footnotesize $\y$};
   \draw[color=neekiRed,variable=x]
     plot function{sin(x)}
     node[right] at (1.5,1.5) {$f(x) = \sin(x)$};
%   \draw[color=neekiBlue,domain=0:5,variable=x]
  \draw[<->]  (0,-3) -- (0,3) node[above] {$y$};
  \draw[<->] (-7, 0) -- (7,0) node[right] {$x$};
  \clip(-5,-5) rectangle (5,5); 
\end{tikzpicture}
\caption{$\sin(x)$}
\end{figure}
 
Here are some equations that are handy to remember. Source \cite{RHBDiffQuickStart}.
\begin{align}
  \deriv{x^n}{x}             & = n{x}^{n-1} \\
  \deriv{{\emph{e}}^{ax}}{x} & = a{\emph{e}}^{ax} \\
  \deriv{\ln(x)}{x}          & = \frac{1}{x} \\
  \deriv{\sin(ax)}{x}        & = (a)\cos(ax) \\
  \deriv{\cos(ax)}{x}        & = (-a)\sin(ax) \\
  \deriv{\tan(ax)}{x}        & = (a)\sec^2(ax) \\
  \deriv{\sec(ax)}{x}        & = (a)\sec(ax)\tan(ax) \\
  \deriv{\csc(ax)}{x}        & = (-a)\csc(ax)\cot(ax) \\
  \deriv{\cot(ax)}{x}        & = (-a)\csc^2(ax) \\
  \deriv{\rsin(\frac{x}{a})}{x}
    & = \frac{1}{\sqrt[2]{{a}^{2} - {x}^{2}}} \\
  \deriv{\rsin(\frac{x}{a})}{x}
    & = \frac{-1}{\sqrt[2]{{a}^{2} - {x}^{2}}} \\
  \deriv{\rsin(\frac{x}{a})}{x}
    & = \frac{a}{\sqrt[2]{{a}^{2} - {x}^{2}}}
\end{align}
For a proof, refer to appendix section \ref{sec:DiffTrigProof},
``Differentiation of Trig Functions Proof''
%-----------------------------------------------------------------------------%
%- Calculus :: Differentiation Maxima and Minima -----------------------------%
%-----------------------------------------------------------------------------%
\chapter{Maxima and Minima}
\label{chap:MaximaAndMinima}
If the function has a negative slope, then $f'(x) < 0$.
If the function has a positive slope, then $f'(x) > 0$.
If the function has a slope of $0$, then $f'(x) = 0$.\footnote{However it is
only instantaneously flat. If you think about it, 0 is slightly bigger than
-0.00001 and slightly smaller than 0.00001 (and you can find arbitrarily
smaller numbers that zero fits between)}
\begin{align}
  f \uparrow \Leftrightarrow f' > 0 \\
  f \downarrow \Leftrightarrow f' < 0
\end{align}

\noindent $f$ has a local maxima at $x=a$ if $f'$ changes sign.

Suppose we were to take the derivative of a derivative. This is called the
\emph{second derivative} and can be written as:
\begin{align}
              f''(x) & = \ldots \nonumber \\
  \frac{d^2 f}{dx^2} & = \ldots \nonumber
\end{align}

\noindent If you want to locate maxima and minima using the second derivative:
$f''$ be aware that the sign change test is more robust than the second
derivative test for local maxima and minima that are undefined.

\noindent In sketching, $f''$, is useful as the sign of $f''$ gives the
concavity (bending up vs down):
\begin{align}
  f'' &> 0 ~ \text{concave up} \\
  f'' &< 0 ~ \text{concave down}
\end{align}

\section{Example}
Suppose the following information has been collected for $f$:

$y' > 0$   at $x<1$
$y' = 0$   at $x=1$
$y' > 0$   at $x=3$
$y' = 0$   at $x=5$
$y' < 0$   at $x>5$

$y'' < 0$   at $x<1$
$y'' = 0$   at $x=1$
$y'' > 0$   at $x=3$
$y'' = 0$   at $x=5$
$y'' < 0$   at $x>5$


\section{Example}
Sketch the curve given by
$y = \frac{x}{x^2+1}$ using $y$, $y'$, $y''$

\begin{align}
  y  &= \frac{x}{x^2+1} \\
  y' &= \frac{1-x^2}{(x^2+1)^2}
\end{align}

\begin{figure}[!hbt]
\begin{tikzpicture}[domain=-5:5, auto,scale=1.5]
  \draw [color=neekiGrey,dash pattern=on 1pt off 1pt]
  (-5,-2) grid (5,2);
  \draw[->,color=gray] (-5,0) -- (5,0);
  \foreach \x in {-5,-4,-3,-2,-1,1,2,3,4,5}
    \draw[shift={(\x,0)},color=black] (0pt,2pt) -- (0pt,-2pt) node[below]
    {\footnotesize $\x$};
  \draw[->,color=gray] (0,-2) -- (0,2);
  \foreach \y in {-2,-1,0,1,2}
    \draw[shift={(0,\y)},color=black] (2pt,0pt) -- (-2pt,0pt) node[left]
    {\footnotesize $\y$};
  \clip (-5,-2) rectangle (5,2);
  %%%%%%%%%%%%%%%%%%%%%%%%%%%%%%%%%%%%%%%%%%%%%%%%%%%%%%%%%%%%%%%%%%%%%%%%%%%%%
  % END GRAPH PREAMBLE
  %%%%%%%%%%%%%%%%%%%%%%%%%%%%%%%%%%%%%%%%%%%%%%%%%%%%%%%%%%%%%%%%%%%%%%%%%%%%%
  \draw[color=neekiBlue,thick,smooth,samples=1000]
    plot[id=maxima_minima_ex2_y] function{ x/(x**2+1) }
    node at (0.5,-0.5) {$y = \frac{x}{x^2+1}$};
  \draw[color=neekiRed,thick,smooth,samples=1000]
    plot[id=maxima_minima_ex2y1] function{ (1-x**2)/(x**2 +1)**2}
    node at (-1.5,0.5) {$y' = \frac{1-x^2}{(x^2+1)^2}$};
  \draw[color=neekiGreen,thick,smooth,samples=1000]
    plot[id=maxima_minima_ex2y11] function{ ( ((x**2 +1)**2)*(-2*x) - (1-x**2)*(4*x*(x**2+1))   )/(((x**2+1)**2)**2)) }   
    node at (-1,1.5) {y'' };
\end{tikzpicture}
\label{fig:maxima_minima_ex2y}
\caption{1 function, and 2 less pleasant functions.}
\end{figure}

\section{Example}
Sketch $y = \frac{2x^2}{x-1}$

\begin{figure}[!hbt]
\begin{tikzpicture}[domain=-5:5, auto,scale=1.5]
  \draw [color=neekiGrey,dash pattern=on 1pt off 1pt]
  (-5,-2) grid (5,2);
  \draw[->,color=gray] (-5,0) -- (5,0);
  \foreach \x in {-5,-4,-3,-2,-1,1,2,3,4,5}
    \draw[shift={(\x,0)},color=black] (0pt,2pt) -- (0pt,-2pt) node[below]
    {\footnotesize $\x$};
  \draw[->,color=gray] (0,-2) -- (0,2);
  \foreach \y in {-2,-1,0,1,2}
    \draw[shift={(0,\y)},color=black] (2pt,0pt) -- (-2pt,0pt) node[left]
    {\footnotesize $\y$};
  \clip (-5,-2) rectangle (5,2);
  %%%%%%%%%%%%%%%%%%%%%%%%%%%%%%%%%%%%%%%%%%%%%%%%%%%%%%%%%%%%%%%%%%%%%%%%%%%%%
  % END GRAPH PREAMBLE
  %%%%%%%%%%%%%%%%%%%%%%%%%%%%%%%%%%%%%%%%%%%%%%%%%%%%%%%%%%%%%%%%%%%%%%%%%%%%%
  \draw[color=neekiBlue,thick,smooth,samples=1000]
    plot[id=maxima_minima_ex2_y] function{ x/(x**2+1) }
    node at (0.5,-0.5) {$y = \frac{x}{x^2+1}$};
  \draw[color=neekiRed,thick,smooth,samples=1000]
    plot[id=maxima_minima_ex2y1] function{ (1-x**2)/(x**2 +1)**2}
    node at (-1.5,0.5) {$y' = \frac{1-x^2}{(x^2+1)^2}$};
  \draw[color=neekiGreen,thick,smooth,samples=1000]
    plot[id=maxima_minima_ex2y11] function{ ( ((x**2 +1)**2)*(-2*x) - (1-x**2)*(4*x*(x**2+1))   )/(((x**2+1)**2)**2)) }   
    node at (-1,1.5) {y'' };
\end{tikzpicture}
\label{fig:maxima_minima_ex3y}
\caption{1 function, and 2 less pleasant functions.}
\end{figure}
%-----------------------------------------------------------------------------%
%- Calculus :: Differentiation -----------------------------------------------%
%-----------------------------------------------------------------------------%
\newpage
\chapter{Differentiation}
\label{sec:Differentiation}
In overly simplified terms, differentiation is the process of taking a curve
on a graph and finding what the gradient of that curve is. There are 4
methods which are useful for MATH130. These are:
\begin{itemize}
  \item Power Method (well, that's what I call it)
  \item Product Rule
  \item Quotient Rule
  \item Chain Rule
\end{itemize}
Because the notion of calculus was a joint effort between several
mathematicians\footnote{Gottfried Leibniz woke up one day and thought ``I'm
going to invent a whole new branch of mathematics to annoy students for the
next few hundred years.'' Approximately 10 years earlier, Sir Isaac Newton
thought ``I know what will really get Leibniz's goat... I'll get the drop on
him with this idea I have.'' Consequently, the two never became friends.}
operating in secret, and without a project manager there are two important
notations.\footnote{there are more, but we don't need to know about them for
MATH130}
\\
The first notation is the ``dash'', ``prime'', or ``Lagrange's'' notation and
appears as such:
\begin{align}
   f(x) & = ... \\
  f'(x) & = ...
  \intertext{Secondly is Leibniz's notation:}
  \\deriv{f(x)}{x} &= ...
  \intertext{Euler's notation: (not so common in MATH130)}
   Df(x) & = ... \\
\end{align}
Each notation has their merits and usefulness; Lagrange's form is neat and
compact for simple derivatives, however Leibniz's notation describe what is
being differentiated and what is in respect to, which is useful for the chain
rule discussed shortly. As such, it is important to be familiar with all the
above forms as they will often be used interchangeably for brevity,
neatness and ease of memorising them. In terms of how you answer a question - 
if there is no stated style of notation, go for what ``looks'' like it works
\footnote{Munner's Law: If it looks wrong it probably is. -- Cliff Munro,
1996 Cranbrook School, Design and Technology teacher. Author's corollary: If it
doesn't look wrong, hopefully it's right.} and is clear and neat. Clear and
neat usually results in the marker understanding what you are on about, so even
if you are wrong, you might get partial marks.
\\
A useful tip, when you are first getting used to differentiation, it may be
handy to use Leibniz's notation and say in your mind what you are
differentiating, and what it is in respect to.
%-----------------------------------------------------------------------------%
%- Calculus :: Differentiation :: Power Method -------------------------------%
%-----------------------------------------------------------------------------%
\section{Power Method}
\label{sec:PowerMethod}
The power method is by far the easiest to understand of all 3 methods, and if
possible, it may be easier to rearrange a part of an equation into index
notation and differentiate that way. This method is not always possible,
but by using the power laws from the series of equations starting with
\ref{eq:IndexLaw_Power0}, sometimes a shortcut can be made, which is why
Section \ref{sec:ExponentialsAndLogarithms} important to know very well.
\\
Put simply, the power method can be understood as ``multiply the base by the
power and subtract one from the power'', and is demonstrated in equation
\ref{eq:DiffPowerMethod} below.
\begin{align}
             f(x) & = {x}^{a} + k \label{eq:DiffPowerMethodKValue} \\
  \deriv{f(x)}{x} & = a{x}^{a-1} \label{eq:DiffPowerMethod}
\end{align}
The value $k$ represents a constant, often just a plain number, though it
doesn't have to be. The important thing about the $k$-value in this example is
that there is no $x$ component. Hence it ``disappears''.\\
Functions may have more than one term, consider the following quadratic:
\begin{align}
  f(x)             & = {x}^{2} + 2xb + {b}^{2} \\
  \deriv{f(x)}{x} & = 2{x}^{1} + 2b \\
                   & = 2(x+b) \label{eq:DiffPowerMethodRespectTo}
\end{align}
Here we are asked to differentiate with respect to $x$ (denoted by the symbol
$\deriv{}{x}$). To do this, we bring the power of ${x}^{2}$ to the front, and
subtract 1 to give $2{x}^{1}$, and we do the same for the term ${2xb}$ by
looking at the invisible power (it's there, we just don't write it out of
laziness!\footnote{Or convenience, neatness, brevity.}): $2*{x}^{1}*b$
%-----------------------------------------------------------------------------%
%- Calculus :: Differentiation :: Product Rules ------------------------------%
%-----------------------------------------------------------------------------%
\section{Product Rule}
\label{sec:ProductRule}
A function $f(x)$ is a product of two functions, $u(x) * v(x)$. For example:
\begin{align}
  f(x) & = {x}^{3} \sin(x) \label{eq:DiffProdEx1}\\
       & = u(x)v(x)
\end{align}
In this case, we can see that $u(x) = {x}^{3}$ and $v(x) = \sin(x)$. While there
is a mathematical proof\footnote{refer to section \ref{eq:ProofOfProductRule} of
appendix} it is not necessary for MATH130. All we need to know is:
\begin{align}
             f(x) & = u(x)*v(x) \\
  \deriv{f(x)}{x} & = u'(x)*v(x) + v'(x)*u(x)
\end{align}
So for \ref{eq:DiffProdEx1}, to find the derivative:
\begin{align}
             f(x) & = {x}^{3}\,\sin(x) \nonumber \\
  \deriv{f(x)}{x} & = \lderiv{({x}^{3})}{x} * \sin(x) + {x}^{3} * \lderiv{\sin(x)}{x} \\
                  & = {3x}^{2}*\sin(x) + {x}^{3}*\cos(x)
\end{align}
This example makes use of a derivative of a trigonometric function. This will be
explored in chapter \ref{chap:DifferentiationOfTrigFunctions}, ``Differentiation
of Trig Functions'' - until then, just ignore the trigonometric part.
%-----------------------------------------------------------------------------%
%- Calculus :: Differentiation :: Quotient Rule ------------------------------%
%-----------------------------------------------------------------------------%
\section{Quotient Rule}
\label{sec:QuotientRule}
A quotient is a division - just like from primary school: $\frac{u}{v}$.
Previously we may have called the $u$ part the numerator, nowadays it is
called the \emph{dividend}, and the $v$ denominator previously called the
denominator is now called the \emph{divisor}, with the \emph{quotient} being
the result.
\\
To do quotient rule differentiation, we are actually using a modified version
of the product rule, however for MATH130 it is only required to think of it as a
separate rule.\footnote{Mathematical proof of this is in section 
\ref{eq:ProofOfQuotientRule} of the appendix.} The rule takes the form of:
\begin{align}
  f(x)
    & = \frac{u}{v} \\
  f'(x)
    & = \frac{\deriv{u}{x}\cdot v - u\cdot \deriv{v}{x}}{{v}^{2}} \label{eq:DiffQuot1}
\intertext{OR}
  f'(x)
  & = \frac{u'v - uv'}{{v}^{2}} \label{eq:DiffQuot2}
\end{align}
Although both \ref{eq:DiffQuot1} and \ref{eq:DiffQuot2} are identical,
\ref{eq:DiffQuot2} is tidier, and may be easier to remember.
Consider the following example:
\begin{align}
  f(x)  & = \frac{{x}^{2}}{4x} \\
  f'(x) & = \deriv{(4x){x}^{2}}{x} - {x}^{2} (\deriv{4x}{x}) \\
  f'(x) & = (4x)2x - {x}^{2}\,(4) \\
  f'(x) & = 8{x}^{2} - 4{x}^{2} \\
  f'(x) & = 4{x}^{2}
\end{align}
%-----------------------------------------------------------------------------%
%- Calculus :: Differentiation :: Chain Rule ---------------------------------%
%-----------------------------------------------------------------------------%
\newpage
\section{Chain Rule}
\label{sec:ChainRule}
The chain rule is useful for differentiation a function $f(x)$ where there are
functions inside of functions (such as $\ln(\sin(x))$). To do this, we break
the function up into it's components, give them some names \footnote{''
\emph{Let}`` is possibly the most important word you will come across in
mathematics. You can use it to redefine stuff if it's too complex and break it
into smaller manageable pieces and put it back together again." -- Chris
Gordon, MATH130 lecturer, Macquarie University, Semester 1 2011} and apply the
chain rule.
\\
The rule takes the form as follows\footnote{using this notation makes it clear as to
what is actually going on}:
\begin{align}
  \deriv{y}{x} & = \deriv{y}{u} \cdot \deriv{u}{x} \\
\end{align}
Here is an example of where we might use the chain rule:\footnote{Note that the
$v$ function above in this example is simply ${u}^{3}$.}\footnote{While we
could use the power method to solve this particular problem in 2 steps, we will
demonstrate chain rule first, and then the power method.}
\begin{align}
  f(x)
    & = {({x}^{2})}^{3}
  \intertext{Let $u = {x}^{2}$ such that}
  f(x)
    & = {u}^{3} \label{eq:DiffChainRuleLet}
  \intertext{The first step in \ref{eq:DiffChainRuleLet} is to identify the
  chain rule, and to name the function inside $u$. It just makes things
  easier this way for most of the time.}
  f'(x)
    & = \deriv{f}{u} \cdot \deriv{u}{x} \\
    & = 3u^2 \cdot 2x
  \intertext{\ref{eq:DiffChainRuleLet} It often helps to rewrite the equation
  in terms of $f(x)$ and $u$, and then to write out the chain rule. Normally we
  are differentiating $f(x)$ with respect to $x$. Here we differentiate $f(x)$
  with respect to $u$ (ie $\deriv{f}{u}$). \emph{THEN} we multiply by
  $\deriv{u}{x}$. The rest is plain old algebra. So.. substitue values back for
  $\deriv{u}{x}$.}
    & = 3u^2 \cdot 2x \nonumber \\
    & = 3(x^2)^2 \cdot 2x \\
    & = 3x^4 \cdot 2x \\
    & = 6x^5
\end{align}
As a point of exercise, here's how much faster it is using the power method:
\begin{align}
  f(x)
    & = {({x}^{2})}^{3} \nonumber \\
  f(x)
    & = {x}^{6}
  \intertext{Now we use the differentiation power method, bring the power out
  the front and reduce the power by one}
  f'(x) & = 6{x}^{5}
\end{align}
Although the power method here took only two steps, it should be noted that
power method cannot be used for all chain rule problems\footnote{in fact, most
of the time the power method won't work because maths teachers are these
hideous evil monsters who despise free time and kittens} - but for easy ones
like this, it is far faster to use power method). The following example cannot
use power method, and we \emph{should} use the chain rule:
\begin{align}
  f(x)
    & = \ln(\sin(x)) \\
  \text{Let } u(x) & = \sin(x) \\
  f'(x)
    & = \deriv{f}{u} \cdot \deriv{u}{x} \\
  \deriv{\sin(x)}{x}
    & = \cos(x) \\
  \deriv{\ln(u)}{u}
    & = \frac{1}{u} \\
  \text{substitute values back:} \\
  f'(x)
    & = \frac{1}{\sin(x)} \cdot \cos(x) \\
    & = \csc(x) \cdot \cos(x) \\
    \text{ or } \\
    & = \frac{\cos(x)}{\sin(x)} \\
    & = \cot(x)
\end{align}
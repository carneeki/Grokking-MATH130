%-----------------------------------------------------------------------------%
%- Calculus :: L'Hopital's Rule ----------------------------------------------%
%-----------------------------------------------------------------------------%
\chapter{L'H\^{o}pital's Rule}
\label{chap:LHopitalsRule}
L'H\^{o}pital's Rule is a way of determining the limit of a function. It has not
been a part of the MATH130 syllabus in recent years, but is incredibly simple
and helps you understand what's going on, so I've included it anyway. Consider
this section optional.

L'H\^{o}pital's Rule states that the limit of a function is defined as the
earliest (or smallest) derivative of the numerator divided by the earliest (or
smallest) derivative of the denominator that can be evaluated as $x$ approaches
the point in question.

This has one important caveat, and that is, that a limit exists. If no limit
exists on the function then L'H\^{o}pital's Rule does not apply.

\section{Example}
\begin{align}
  \lim_{x \to \infty} \frac{6x^2 -x}{6x^2 -5x} &= \\
\intertext{Obviously, we cannot evaluate when $x=\infty$, so we take the
derivative of both numerator, and denominator}
  &= \frac{12x -1}{12x-5} \\
\intertext{We still cannot evaluate this, so we repeat}
  &= \frac{12}{12} \\
  &= 1 \\
\intertext{A limit of $1$, shows that the function $\frac{6x^2-x}{6x^2-5x}$ is
asympotic at $x=1$.}
\end{align}
\begin{figure}[!hbt]
\begin{tikzpicture}[domain=-5:5, auto,scale=1]
  \draw [color=neekiGrey,dash pattern=on 1pt off 1pt]
  (-5,-2) grid (5,2);
  \draw[->,color=gray] (-5,0) -- (5,0);
  \foreach \x in {-5,-4,-3,-2,-1,1,2,3,4,5}
    \draw[shift={(\x,0)},color=black] (0pt,2pt) -- (0pt,-2pt) node[below]
    {\footnotesize $\x$};
  \draw[->,color=gray] (0,-2) -- (0,2);
  \foreach \y in {-2,-1,0,1,2}
    \draw[shift={(0,\y)},color=black] (2pt,0pt) -- (-2pt,0pt) node[left]
    {\footnotesize $\y$};
  \clip (-5,-2) rectangle (5,2);
  %%%%%%%%%%%%%%%%%%%%%%%%%%%%%%%%%%%%%%%%%%%%%%%%%%%%%%%%%%%%%%%%%%%%%%%%%%%%%
  % END GRAPH PREAMBLE
  %%%%%%%%%%%%%%%%%%%%%%%%%%%%%%%%%%%%%%%%%%%%%%%%%%%%%%%%%%%%%%%%%%%%%%%%%%%%%
  \draw[color=neekiBlue,thick,smooth,samples=100,domain=-5:0.8]
    plot[id=lhopital_ex1a] function{ (6*x**2-x)/(6*x**2-5*x) }
    node at (2,-1) {$y=\frac{6x^2-x}{6x^2-5x}$};
  \draw[color=neekiBlue,thick,smooth,samples=100,domain=0.9:5]
    plot[id=lhopital_ex1b] function{ (6*x**2-x)/(6*x**2-5*x) };
  \draw[color=neekiRed,dash pattern=on 2pt off 2pt,thick]
    (-5,1) -- (5,1);
\end{tikzpicture}
\label{fig:lhopital_ex1}
\caption{Limit as $x\to\infty$}
\end{figure}

\section{Example}
\begin{align}
  \lim_{x \to \infty} \frac{-2x^3+5x^2}{-x^3} &= \\
\intertext{We cannot evaluate when $x=\infty$, so we take the
derivative of both numerator, and denominator}
  &= \frac{6x^2+10x}{-3x^2} \\
\intertext{We still cannot evaluate at $x=\infty$, so we repeat}
  &= \frac{12x+10}{-6x} \\
\intertext{We still cannot evaluate at $x=\infty$, so we repeat}
  &= \frac{12}{-6} \\
  &= -2 \\
\intertext{A limit of $-2$, shows that the function is asympotic at
$x=-2$.}
\end{align}
\begin{figure}[!hbt]
\begin{tikzpicture}[domain=-10:10, auto,scale=1,x=0.75cm]
%  \draw [color=neekiGrey,dash pattern=on 1pt off 1pt]
%  (-10,-5) grid (10,2);
  \draw[->,color=gray] (-10,0) -- (10,0);
  \foreach \x in {-10,-9,-8,-7,-6,-5,-4,-3,-2,-1,1,2,3,4,5,6,7,8,9,10}
    \draw[shift={(\x,0)},color=black] (0pt,2pt) -- (0pt,-2pt) node[below]
    {\footnotesize $\x$};
  \draw[->,color=gray] (0,-5) -- (0,2);
  \foreach \y in {-5,-4,-3,-2,-1,0,1,2}
    \draw[shift={(0,\y)},color=black] (2pt,0pt) -- (-2pt,0pt) node[left]
    {\footnotesize $\y$};
  \clip (-10,-5) rectangle (10,2);
  %%%%%%%%%%%%%%%%%%%%%%%%%%%%%%%%%%%%%%%%%%%%%%%%%%%%%%%%%%%%%%%%%%%%%%%%%%%%%
  % END GRAPH PREAMBLE
  %%%%%%%%%%%%%%%%%%%%%%%%%%%%%%%%%%%%%%%%%%%%%%%%%%%%%%%%%%%%%%%%%%%%%%%%%%%%%
  \draw[color=neekiBlue,thick,smooth,samples=100,domain=-10:-0.1]
    plot[id=lhopital_ex2a] function{ (2*x**3+5*x**2)/(-x**3) }
    node at (2,-1) {$y=\frac{-2x^3+5x^2}{-x^3}$};
  \draw[color=neekiBlue,thick,smooth,samples=100,domain=0.1:10]
    plot[id=lhopital_ex2b] function{ (2*x**3+5*x**2)/(-x**3) };
  \draw[color=neekiRed,dash pattern=on 2pt off 2pt,thick]
    (-10,-2) -- (10,-2);
\end{tikzpicture}
\label{fig:lhopital_ex2}
\caption{Limit as $x\to\infty$}
\end{figure}


\section{Example}
\begin{align}
  \lim_{x \to \infty} \frac{-3x^3-6x^2}{6x^3-4x^2} &= \\
\intertext{We cannot evaluate when $x=0$, so we take the
derivative of both numerator, and denominator}
  &= \frac{-9x^2-12x}{18x^2-8x} \\
\intertext{We still cannot evaluate at $x=0$, so we repeat}
  &= \frac{-18x-12}{36x-8} \\
\intertext{At this point, we can evaluate at $x=0$,}
  &= \frac{-18(0)-12}{36(0)-8} \\
  &= \frac{-12}{-8} \\
  &= \frac{3}{2}
\end{align}
A limit of $\frac{3}{2}$, shows that when $x=0$, it cuts the $y$-axis, but in
this particular case, is not asymptoic at this point. If we were to take the
limit at $x\to\infty$, we would find it is $-\frac{1}{2}$, indicated in red
below.

\begin{figure}[!hbt]
\begin{tikzpicture}[domain=-10:10, auto,scale=1,x=0.75cm]
%  \draw [color=neekiGrey,dash pattern=on 1pt off 1pt]
%  (-10,-5) grid (10,2);
  \draw[->,color=gray] (-10,0) -- (10,0);
  \foreach \x in {-10,-9,-8,-7,-6,-5,-4,-3,-2,-1,1,2,3,4,5,6,7,8,9,10}
    \draw[shift={(\x,0)},color=black] (0pt,2pt) -- (0pt,-2pt) node[below]
    {\footnotesize $\x$};
  \draw[->,color=gray] (0,-5) -- (0,2);
  \foreach \y in {-5,-4,-3,-2,-1,0,1,2}
    \draw[shift={(0,\y)},color=black] (2pt,0pt) -- (-2pt,0pt) node[left]
    {\footnotesize $\y$};
  \clip (-10,-5) rectangle (10,2);
  %%%%%%%%%%%%%%%%%%%%%%%%%%%%%%%%%%%%%%%%%%%%%%%%%%%%%%%%%%%%%%%%%%%%%%%%%%%%%
  % END GRAPH PREAMBLE
  %%%%%%%%%%%%%%%%%%%%%%%%%%%%%%%%%%%%%%%%%%%%%%%%%%%%%%%%%%%%%%%%%%%%%%%%%%%%%
  \draw[color=neekiBlue,thick,smooth,samples=100,domain=-10:0.6]
    plot[id=lhopital_ex3a] function{ (-3*x**3-6*x**2)/(6*x**3-4*x**2) }
    node at (2,1) {$y=\frac{-3x^3-6x^2}{6x^3-4x^2}$};
  \draw[color=neekiBlue,thick,smooth,samples=100,domain=0.8:10]
    plot[id=lhopital_ex3b] function{ (-3*x**3-6*x**2)/(6*x**3-4*x**2) };
  \draw[color=neekiPurple,dash pattern=on 2pt off 2pt,thick]
    (-10,1.5) -- (10,1.5);
  \draw[color=neekiRed,dash pattern=on 2pt off 2pt,thick]
    (-10,-0.5) -- (10,-0.5);
\end{tikzpicture}
\label{fig:lhopital_ex3}
\caption{Limit as $x\to 0$ (purple) and $x \to \infty$ (red).}
\end{figure}
%-----------------------------------------------------------------------------%
%- Calculus :: Differentiation of Trigonometric Functions --------------------%
%-----------------------------------------------------------------------------%
\chapter{TODO: Newton's Method}
\label{chap:NewtonsMethod}
\begin{align}
  x_1 = x_0 - \frac{f(x_0)}{f'(x_0)}
\end{align}
Newton's Method is used to find successively better approximations of the roots
of a function.

We take a value of $x$, $x_0$ and subtract the function at $x_0$ divided by the
derivative of that function at $x_0$. The result, $x_1$ is an answer close to
the roots of that function.

What we can do next, is put $x_1$ back into the same process and get a better
approximation of the function. If we repeat this over and over, we will either
get an exact value of the roots of the function, or we will get pretty darn
close to it.

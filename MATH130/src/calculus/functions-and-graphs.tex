\chapter{Calculus}
\label{chap:Calculus}
%-----------------------------------------------------------------------------%
%- Calculus :: Functions & Graphs --------------------------------------------%
%-----------------------------------------------------------------------------%
\chapter{Functions \& Graphs}
\label{chap:FunctionsAndGraphs}
If you are a computer programmer, the best way to think of a function in maths
is the same way you think of a function in a functional programming language. If
you are not a computer programmer, perhaps the best way to think of a function
is like a little machine that takes a number, it does something to that number,
and it displays an output.\footnote{Congratulations, you are now thinking like a
programmer as well as a mathematician!} Here is an example of a function that
simply ``doubles'' the input:
\begin{align}
  f(x) & = 2x
  \intertext{Here is what happens when we input the number $5$ into our
  function...}
  f(5)  & = 2(5) \\
        & = 10
  \intertext{And now the number $-5$:}
  f(-5) & = 2(-5) \\
        & = -10
  \intertext{Functions can be (and often are) more complex, here's a
  quadratic function:}
  f(a)  & = {(a+3)}^{2} \\
        & = (a+3)\,(a+3) \\
        & = {a}^{2} + 6a + 9
  \intertext{And if we substitute $5$ in for $a$ we get:}
  f(5)  & = {5}^{2} + 6(5) + 9\\
        & = 25 + 30 + 9 \\
        & = 64
  \intertext{For completeness, if we substitute $-5$ in for $a$ we get:}
  f(-5) & = {-5}^{2} + 6(-5) + 9\\
        & = 25 -6(5) + 9 \\
        & = 4
\end{align}
When it comes to graphing functions, you can rename your $y$ axis to equal
$f(x)$, so graphing your function is now the same as before graphs with one
additional bonus: now you can let $x$ be an abitrary number\footnote{This is
called a \emph{independent variable}} as part of the function. Figure
\ref{fig:FuncGraphLinear} shows a function, $f(x) = {x}$.
\\
Types of graphs that can be encountered in MATH130:
\begin{enumerate}
  \item Line \ref{fig:FuncGraphLinear}
  \item Parabola \ref{fig:FuncGraphParabola}
  \item Hyperbola \ref{fig:FuncGraphHyperbola}
  \item Cubic \ref{fig:FuncGraphCubic}
  \item Absolute \ref{fig:FuncGraphAbsolute}
\end{enumerate}

A function has a \emph{domain} which is all the acceptable values of $x$ as
inputs that the function can use. Some examples:

\begin{align}
   f(x) & = x^{2} \nonumber \\
   domain: { 0 \leq x \geq 0 } \\
   \\
   f(x) & = \frac{1}{x} \nonumber \\
   domain: { x \neq 0 } \\
   \\
   f(x) & = \sqrt{x} \nonumber \\
   domain: { x \geq 0 } \\
   \\
   f(x) & = \sqrt{x-2} + \sqrt{5 + x} \nonumber \\
   domain: (x-2) & \geq 0 \\
      \therefore & \geq 2 \\
      \and \\
   domain: 5 + x & \geq 0 \\
      \therefore & \geq -5 \\
   and so:\\
   {x: x \geq 2}
\end{align}
The last equation demonstrates that the domain is the more restrictive of
the conditions of the two ``sub-domains'' of each square root portion of the
function.

\section{Arithmetic With Functions}
\label{sec:ArithmeticWithFunctions}

Suppose we have:
\begin{align}
  f(x) & = x^{2} + 1 \\
  g(x) & = \frac{1}{1-x} \\
  \intertext{and}
  h    & = f(x) + g(x) \\
  \intertext{then}
       & = x^{2} + 1 + \frac{1}{1-x} \\
  \intertext{however, if}
  i    & = f(x) \times g(x) \\
  \intertext{then}
       & = (x^{2} + 1) \times (\frac{1}{1-x}) \\
  \intertext{however, if}
  j    & = f \circ g \\
  \intertext{then}
       & = j(x) = f(g(x)) \\
       & = (\frac{1}{1-x})^{2} + 1
\end{align}
\quote{``Remember to ``algebra'' the function to minimise them and to see if
they equal a simpler equation''}\footnote{according to Gareth Richardson}

%
\clearpage
\section{Linear Functions}
\label{sec:LinearFunctions}
\begin{figure}[!hbt]
\label{fig:FuncGraphLinear}
\begin{tikzpicture}[scale=0.5]
  \label{fig:FuncGraphLine}
  \draw[<->]  (-3.9, -1.9) -- (3.9,9.9) node[right] {$f(x) = mx + b$};
  \draw[<->]   (0.0, -1.9) -- (0.0,9.9) node[above] {$f(x)$};
  \draw[<->]  (-3.9,  0.0) -- (3.9,0.0) node[right] {$x$};
\end{tikzpicture}
\caption{A linear function: $f(x) = mx + b$}
\end{figure}
%
Linear function in the \emph{General Form} $Ax + By + C= 0$. It may also take
the handy \emph{Slope-intercept form} $ f(x) = mx + b $. This is useful because
the gradient of the line cane be read straight from the equation, and is just
a rearrangement of the general form. 
\begin{table}[!hbt]
\label{tab:PartsOfALinearFunction}
\begin{tabularx}{\linewidth}{| l X |}
  \hline
  \multicolumn{2}{|l|}{Where:} \\
  \hline \hline
  m & represents the gradient\\
  x & is the independent variable\\
  y & is the dependent variable\\
\hline
\end{tabularx}
\end{table}

If given only 2 points and we are to find the equation of the line:
\begin{enumerate}
  \item we need to determine the gradient
  \item we need to substitute the x,y values in to form the equation
\end{enumerate}

Determining the gradient is done using the formula:
\begin{align}
  \frac{y_{2} - y_{1}}{x_{2} - x_{1}} & = m
\end{align}

Example: we are given the points (7,1), (2,5) and need to find the equation of
the line connecting these points.
\begin{align}
  \frac{5 - 1}{2 - 7} & = \frac{4}{-5}\\
    & = -\frac{4}{-5}
\end{align}

If we want to find a parallel line passing through specific points, remember
that the gradient ($m$) must be the same in both equations, and we must
subtitute the $x$ and $y$ values for the specific points into the new arbitrary
equation to solve for the new equation:

Example: find an equation for the line through $(3,4)$ and parallel to the line
through $(7,1)$ and $(2,5)$ from out previous example:
\begin{align}
  \intertext{let}
  m & = -\frac{4}{5} \nonumber \\
  \intertext{substitute x,y values into slope-intercept equation}
  y & = mx + b \nonumber \\
  4 & = -frac{4}{5}\times3 + b
  \intertext{and solve for b}
  4 + \frac{4}{5}\times3 & = b \\ 
  4 + \frac{12}{5}       & = b \\
  \frac{32}{5}           & = b \\
  \therefore y & = -\frac{4}{5}x + \frac{32}{5}
\end{align}
%
\clearpage
\section{Parabolic Functions}
\label{sec:ParabolicFunctions}
For an introduction to parabolas, it is highly recommended to read chapter
\ref{chap:Parabolas}, ``Parabolas''.
\clearpage
\section{Hyperbolic Functions}
\begin{figure}[!htb]
\label{fig:FuncGraphHyperbola}
\begin{tikzpicture}[samples=50,scale=1]
   \draw[color=neekiRed,domain=-5:-0.2,variable=x]
     plot[id=hyperbx.neg] function{1/x}
     node[right] {$f(x) = \frac{1}{x}, x \in \{ -5,-\frac{1}{5} \} $};
   \draw[color=neekiBlue,domain=0.2:5,variable=x]
   plot[id=hyperbx.pos] function{1/x}
     node[right] {$f(x) = \frac{1}{x}, x \in \{ \frac{1}{5},5 \} $};
  \draw[<->]  (0.0,-5.0) -- (0.0,5.0) node[above] {$f(x)$};
  \draw[<->] (-5.0, 0.0) -- (5.0,0.0) node[right] {$x$};
\end{tikzpicture}
\caption{A hyperbolic function: $f(x) = \frac{1}{x}$}
\end{figure}
A hyperbolic function in the form $ f(x) = \frac{1}{x}$
\begin{table}[!hbt]
\label{tab:PartsOfAHyperbolicFunction}
\begin{tabularx}{\linewidth}{| l X |}
  \hline
  \multicolumn{2}{|l|}{Where:} \\
  \hline \hline
  x & is the independent variable\\
  y & is the dependent variable\\
\hline
\end{tabularx}
\end{table}
%
\clearpage
\section{Cubic Functions}
\begin{figure}[!hbt]
\label{fig:FuncGraphCubic}
\begin{tikzpicture}[scale=1]
  \def\xmin{-5}
  \def\xmax{5}
  \def\ymin{-5}
  \def\ymax{5}
  \clip
    (\xmin,\ymin) rectangle (\xmax,\ymax);
  \draw[color=neekiBlue, <->, variable=x]
    plot[id=parabola, raw gnuplot, smooth]
    function{
      set xrange [\xmin:\xmax];
      set yrange [\ymin:\ymax];
      plot x**3;}
    node[right] {$f(x) = {x}^{3}, x \in \{ -5,5 \} $};
  % grid
  \draw[very thin, color=black!30, ystep=1, xstep=1]
    (\xmin,\ymin) grid (\xmax,\ymax);
  % x-axis
  \draw[<->]
    (\xmin,0) -- (\xmax,0) node[right] {$x$};
  % y-axis
  \draw[<->]
    (0,\ymin) -- (0,\ymax) node[above] {$f(x)$};
\end{tikzpicture}
\caption{A cubic function: $f(x) = {x}^{3}$}
\end{figure}
A cubic function in the form $ f(x) = mx^3 + c$
\begin{table}[!hbt]
\label{tab:PartsOfACubicFunction}
\begin{tabularx}{\linewidth}{| l X |}
  \hline
  \multicolumn{2}{|l|}{Where:} \\
  \hline \hline
  m & represents a component of the gradient (covered more in section
  \ref{sec:Differentiation} ``Differentiation''). \\ x & is the independent
  variable\\ y & is the dependent variable\\
  x & is the independent variable\\
  y & is the dependent variable\\
\hline
\end{tabularx}
\end{table}
%
\clearpage
\section{Absolute Value Functions}
\begin{figure}[!hbt]
\label{fig:FuncGraphAbsolute}
\begin{tikzpicture}[scale=1]
 \def\xmin{-5}
 \def\xmax{5}
 \def\ymin{-2}
 \def\ymax{5}
 \clip
   (\xmin,\ymin) rectangle (\xmax,\ymax);
 \filldraw[fill=blue,fill opacity=0.15, color=neekiBlue, <->, variable=x]
   plot[id=parabola, raw gnuplot, smooth]
   function{
     set xrange [\xmin:\xmax];
     set yrange [\ymin:\ymax];
     plot abs(x);}
   node[right] {$f(x) = |x|, x \in \{ -5,5 \} $};
 % grid
 \draw[very thin, color=black!30, ystep=1, xstep=1]
   (\xmin,\ymin) grid (\xmax,\ymax);
 % x-axis
 \draw[<->]
   (\xmin,0) -- (\xmax,0) node[right] {$x$};
 % y-axis
 \draw[<->]
   (0,\ymin) -- (0,\ymax) node[above] {$f(x)$};
\end{tikzpicture}
\caption{A function of absolute value: $f(x) = |x|$}
\end{figure}
An absolute function in the form $ f(x) = |x|$
\begin{table}[!hbt]
\label{tab:PartsOfAnAbsoluteFunction}
\begin{tabularx}{\linewidth}{| l X |}
  \hline
  \multicolumn{2}{|l|}{Where:} \\
  \hline \hline
  m & represents a component of the gradient (covered more in section
  \ref{sec:Differentiation} ``Differentiation''). \\ x & is the independent
  variable\\ y & is the dependent variable\\
  x & is the independent variable\\
  y & is the dependent variable\\
\hline
\end{tabularx}
\end{table}
\clearpage
%-----------------------------------------------------------------------------%
%- Calculus :: Differentiation of Trigonometric Functions --------------------%
%-----------------------------------------------------------------------------%
\chapter{TODO: Integration}
\label{sec:Integration}
tl;dr: Integration is the reverse process of Differentiation.

If we were to delve more into what integration is, then we could describe it as
finding the area ``underneath a curve'' (more accurately, between the curve and
the x-axis).

There are many different ways of doing this, and MATH130 requires we know 2 of
them:
\begin{description}
  \item[Simpson's Rule]
  \item[Trapezoidal Rule]
\end{description}

There are two types of integration:
\begin{description}
  \item[Anti-derivatives] or \emph{indefinite integrals}

  \item[Definite Integrals]
\end{description}
The principles behind indefinite and definite integrals are the same, however,
with definite integrals, you have a range for which you are integrating your
function, where as indefinite integrals have no range (beyond the complete range
of the function itself).
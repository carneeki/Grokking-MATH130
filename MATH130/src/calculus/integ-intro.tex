%-----------------------------------------------------------------------------%
%- Calculus :: Differentiation of Trigonometric Functions --------------------%
%-----------------------------------------------------------------------------%
\chapter{TODO: Integration}
\label{sec:Integration}
tl;dr \emph{Integration is the reverse process of Differentiation.}

Integration is as looking at the \emph{accumulated change}, often by summing (or
subtracting) of a quantity with respect to another quantity.

If we were to delve more into what integration is, then we could describe it as
finding the area ``underneath a curve'' (more precisely, between the curve and
the x-axis).

There are many different ways of doing this, and MATH130 requires we know 2 of
them:
\begin{description}
  \item[Riemann Sums]
  \item[Simpson's Rule]
  \item[Trapezoidal Rule]
\end{description}

There are two types of integration:
\begin{description}
  \item[Anti-derivatives] or \emph{indefinite integrals}
  \item[Definite Integrals]
\end{description}
The principles behind indefinite and definite integrals are the same, however,
with definite integrals, you have a range for which you are integrating your
function, where as indefinite integrals have no range (beyond the complete range
of the function itself).

\section{Riemann Sums}
\begin{enumerate}
  \item Draw a bunch of rectangles from the $x$-axis to the function
such that the top left corner touches the function. This left hand corner is called a
\emph{Left-hand Riemann sum.}

  \item Now draw the same, but with the right hand corner, a ``Right-hand
 Riemann sum''.
\end{enumerate}

For the function $2^x$ divided into 4 rectangles, the area can be calculated as

$L_4 = \frac{1}{2} \left(2^1 + 2^{1.5} + 2^{2} + 2^{2.5}\right) \approx 7.24$
and
$R_4 = \frac{1}{2} \left(2^{1.5} + 2^{2} + 2^{2.5} + 2^{3}\right) \approx 10.24$

\noindent Our intuition tells us to take the mean of these two numbers, however
the curve is not quite straight, so the mean would be slightly larger. This mean
($8.74$). It is pretty close though, and the most you can be wrong by is the
interval of the gap divide by two\ldots That is, $\frac{R_4-L_4}{2} =
\frac{3}{2} = 1.5$.

\noindent Suppose we were to draw more intervals, $L_5$ and $R_5$ would be
closer. $L_{500}$ and $R_{500}$ would be closer again. The error converges to
zero with more rectangles.

\noindent More generally,
\begin{align}
  L_n &= 2^{1\cdot \Delta x} +2^{2\cdot \Delta x} +2^{3\cdot \Delta x} +
  \ldots + 2^{(n-1)\cdot \Delta x} \\
  L_n &= f(1)\Delta x + f(1+\Delta x)\Delta x + \ldots + f(1+(n-1)\Delta
  x)\Delta x \\
  L_n &= \sum_{i=0}^{n-1} f(1 +i\Delta x)\Delta x
\intertext{The exact answer can be given by:}
  A &= \lim_{n \to \infty} \sum_{i=0}^{n-1} f(1+i \Delta x) \Delta x \\
    &= \int_{1}^{3} f(x) dx
\end{align}

\noindent This formula is called ``the definite integral of f from $x=1$ to
$x=3$''.
Where:
\begin{table}[!hbt]
\label{tab:PartsOfAnIntegral}
\begin{tabularx}{\linewidth}{| l X |}
\hline
\multicolumn{2}{|l|}{The convention I will use here is:} \\
\hline \hline
$\int$ & is the integral representing the \emph{sum} \\
$\int_a$ & $a$ is our starting point \\
$\int_{}^b$ & $b$ is our finishing point \\
$f(x)$ & is the function to be integrated \\
$dx$   & is what we are integrating with respect to \\
\hline
\end{tabularx}
\caption{Parts of integrating a function}
\end{table}

\section{General form of a Riemann Sum}
\begin{align}
  L_n &= f(a)\Delta x + f(a + 1 \Delta x)\Delta x + f(a + 2 \Delta x)\Delta x +
  \ldots + f(a + (n-1)\Delta x)\Delta x \\
      &= \sum_{i=0}^{n-1} f(a+i \Delta x)\Delta x \\
      &= \sum \ldots \\
  L_n & \to A ~\text{as} ~ n \to \infty \\
  A &= \lim_{n \to \infty} f(a+i \Delta x)\Delta x \\
    &= \int_a^b f(x) dx
\end{align}

\section{Technique of Integration: Undoing the Chain Rule}
Recall the chain rule:
\begin{align}
  (f \circ g)'(x) &= f'(g(x)) \cdot g'(x) \nonumber \\
\intertext{By reversing we have}
  \int f'(g(x))\cdot g'(x) \nd{x} &= (f \circ g)(x) \\
\intertext{Let}
  f'(g(x)) &= f'(u) \\
  g'(x)\nd{x} &= \nd{u} \\
\intertext{Such that}
\int f'(g(x))\cdot g'(x) \nd{x} &= \\
  &= \int f'(u)\nd{p} \\
  &= f(g(x))
\intertext{This works because $p$ is an inside function, and we can
differentiate it:}
  \deriv{p}{x} &= g'(x) \\
  \nd{p} &= g'(x)\nd{x}
\end{align}

\subsection{Example}
\begin{align}
  \int \left(15x^2 + 4\right)\left(15x^3+4x+1\right)^8\nd{x} &=
  \intertext{Let the inside function, u}
  u &= 5x^3+4x+1 \\
  \intertext{such that}
  \deriv{u}{x} &= 15x^2 + 4 \\
  \nd{u} &= \left(15x^2 + 4\right)\nd{x}
  \intertext{and}
  \int \left(15x^2 + 4\right)\left(15x^3+4x+1\right)^8\nd{x} &= \\
    &= \int u^8\nd{u} \\
    &= \frac{u^9}{9} \\
  \intertext{then substitute back}
    &= \frac{\left(5x^3+4x+1\right)^9}{9}
\end{align}

\subsection{Example}
\begin{align}
  \int \frac{1}{8x-3}\nd{x} &= \int \left(8x-3\right)^{-1}
  \intertext{It is handy to remember:}
    \int \frac{1}{x} \nd{x} &= \log(|x|) \\
  \intertext{let u}
  u &= 8x -3 \\
  \deriv{u}{x} & = 8 \\
  \intertext{thing}
  &= \frac{1}{8} \int \frac{1}{u} \nd{u} \\
  &= \frac{1}{8} \log(|u|) \\
  &= \frac{1}{8} \log(|8x-3|)
\end{align}

Rule to memerise:
\begin{align}
  \int f'(x) e^{f(x)} \nd{x} &= e^{f(x)}
  \intertext{we can get to this via $u$ substition, let}
  u &= f(x) \\
  \deriv{u}{x} &= f'(x) \\ 
  \nd{u} &= f'(x)\nd{x} \\
  \int e^u \nd{x}
\end{align}

\begin{align}
  \int \frac{f'(x)}{f(x)} \nd{x} &= \log(|f(x)|) \\
  \intertext{By:}
   \int \frac{u'}{u} \nd{u} \\
  &= \int \frac{1}{u} \nd{u}
  \intertext{Where}
  u &= f(x)
\end{align}
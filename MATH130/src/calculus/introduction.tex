%-----------------------------------------------------------------------------%
%- Calculus :: Introduction --------------------------------------------------%
%-----------------------------------------------------------------------------%
\chapter{Introduction to Calculus}
\label{chap:IntroductionToCalculus}
In a nutshell calculus is about measuring small changes. There are 4 major
topics in calculus, with the first two of interest to MATH130 students; the
latter two being more advanced topics for down the road.
\begin{description}
  \item[Differential Calculus] Studying the rates at which things change at
  specific points (most often, the rate at which something changes with respect
  to time - eg acceleration can be differentiated to determine speed of an
  object at a given time). This is concerned with the instantaneous rate of
  change. The derivative (gradient) is able to show this. 
  
  \item[Integral Calculus] The reverse of differentiation: taking a series of
  changes and turning this into a different metric (eg, taking a series of
  points of speed and turning this into the acceleration of an object). This is
  concerned with the accumulation of metrics. The area between a curve and the
  x-axis is able to show this.
  
  \item[Multivariable Calculus] Extends the previous two types of calculus by
  allowing one to differentiate or integrate with respect to multiple variables.
  
  \item[Vector Caclulus] Concerns itself with differentiating and integrating
  things called vector fields in 3D space. It is a subset of multivariable
  calculus in that a vector is a set of points in 3D that define a ray (kind of
  like a line, but with 2 end points) starting at 0,0,0 and ending at some x,y,z
  coordinate in 3D space.
\end{description}
There are several principles which are required for differential and integral
calculus:
\begin{description}
  \item[Limits] A limit is basically a really small value that represents the
  difference between the two inputs of a function. We say it is
  \emph{``sufficiently close''} with the result is arbitrarily close enough to
  be deemed the closest we can measure.
  
  \item[Derivatives] Consider a non-linear function, $f(x)$. The derivative is
  the gradient of the function at a given point. The gradient changes depending
  on the $x$-value we supply; so we find a derivative at a certain point. This
  is done by the process of differentiation; and we then substitute the value of
  $x$ in for that point to determine\footnote{derive??} the gradient.
  
  \item[Fundamental theorem] States that differentiation and integration are
  inverse operations - that is, one will undo the other. There are two parts to
  integration; definite integrals and indefinite integrals (sometimes called
  \emph{antiderivatives}). These will be covered in whole sections unto
  themselves.
\end{description}

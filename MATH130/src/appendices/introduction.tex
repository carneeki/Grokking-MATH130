%-----------------------------------------------------------------------------%
%- Acknowledgments -----------------------------------------------------------%
%-----------------------------------------------------------------------------%
\chapter{Copyright and Terms of Use, and Other Bits}
\label{chap:Copyright}
\section{Copyright}
This work has been authored mostly by Adam Carmichael,
\url{adam.carmichael@students.mq.edu.au} and is mostly his copyright material,
except where noted.

Use of the source of this work is licensed under a Creative Commons
Attribution NonCommercial NoDerivs 3.0 Unported License. The full details of
this license are available from
\url{http://creativecommons.org/licenses/by-nc-nd/3.0/}.

I chose a NoDerivs suffix as it is trivial to take this work, and push
``compile'' button in any \LaTeX~editor and generate a textbook to take to a
publisher. It would be unfair (on myself, as well as everyone who helped) to
allow that to happen. Should you wish to generate a textbook, please contact the
author to discuss textbook licensing.

If you wish to make ammendments to details, by all means, fork the code, and
send me a patch containing your changes, or a ``pull request'' (or both). You
will get a mention in the acknowledgements, and if a publisher wants a textbook
you will also get a reward commensurate with the changes you provide.

\section{Terms and conditions of use}
This work is an open source project. It strives to be technically correct and
accurate, though, it may not be.

\section{A cautionary word}
I've failed MATH130 at least twice. I've been pissed with the Maths
department. I've written most of this book whilst ropable at them for not doing
it themselves. This may not be the best resource, but they haven't really done a
good enough job of organizing their resources, so resources are limited. Please
attend lectures, tutorials and practicals. Go ask your teachers, tutors,
professors, lecturers, and friends questions and be active in classes. It is
said the only way to learn math is to do math. If you say ``screw math'', it
\emph{will} screw you\footnote{And it might screw you anyway\ldots}.

\section{Typesetting}
This book has been typeset in \LaTeX~using a Lenovo X201 Tablet running
Microsoft Windows 7, Eclipse Indigo, TeXlipse and MikTex 2.9 64-bit beta. I've
been drinking copious amounts of energy drinks and coffee; if you would like to
become a sponsor and have your name mentioned here, let me know.

The source is presently available from
\url{https://github.com/carneeki/Grokking-MATH130}. You should be able to find
updates of the PDF at \url{http://goo.gl/8GHJj}. The PDF may not always match
the latest source, but you can always compiled the PDF yourself.

Additionally, most URLs in the PDF should be clickable. URLs in footnotes may
not be; this is something that needs to be debugged.

\section{Introduction}
\label{sec:Introduction}
%A long long time ago, in an office far far away, a bright mathematician decided
%to come up with a method for making sure MATH students at Macquarie had no
%social life by ensuring our mathematics was up to scratch. Unfortunately, some
%of us decided to turn mathematics into a social affair by interpreting that
%bloke's notes in an easy manner.
In 1999 a pair of elite mathematicians\footnote{Chen and Duong} decided to
write some notes for a subject they knew plenty about. The notes proved
difficult to understand by some and could have been made easier by means of an
introduction in plain simple English. Today they survive as reference documents
on Rutherglen. If you can find them, and you read them, and you have this
guide, then maybe you can pass MATH130.\\
\emph{(To be read whilst playing the introduction to the A-Team).}\\
\\
Typically the syllabus is broken into two streams, calculus and algebra. This
gives rise to certain problems if algebra falls behind calculus because
there are prerequisites in algebra to solving some calculus problems. As such,
these notes will be arranged such that the algebra material is covered first.
%-----------------------------------------------------------------------------%
% Format and styling in this file originally created by 
% Carl E. Svensson 2010, updated by Adam J. Carmichael 2011
%-----------------------------------------------------------------------------%
%
%- Document Makeup -----------------------------------------------------------%
%- (01) Notes from template author
%- (02) Document Class and Options
%- (03) Standard package includes and options
%- (04) Custom Definitions and Alterations
%- (05) Custom Commands
%- (06) Document title and other metadata
%- (07) Start Document Content
%- (07a) Misc Config 
%
%
%- (01) Notes from carneeki@ -------------------------------------------------%
% NEATNESS:
%  Please keep the TeX neat. Best ways to do this:
%  (01) Don't indent
%  (02) Keep inside of 80 characters (it makes for nicer editing on small
%       laptops).
%  (03) Avoid whitespace between \section{} and other document elements. We
%       have %%%% comments for a reason!
%  (04) Use 2 (that's TWO) space characters to indent. NEVER use tab unless
%       your editor converts to to space chars.
%  (05) Maintain customisations in their respective sections.
%  (06) Comment everything. Bandwidth and diskspace are cheap these days, and
%       TeX compresses pretty nice. Anything else is the BAD kind of laziness
%       on your part.
%
% MULTILINE EQUATIONS:
%  Use \begin{align} instead of \begin{eqnarray}...
%  Details as to why are found at (tl;dr : it's just better...):
%  http://texblog.net/latex-archive/maths/eqnarray-align-environment/
% 
% BIBLIOGRAPHY: 
%  -> URLS: to generate the GUID for a reference that is for a URL, paste
%     the URL into goo.gl and then take only the suffix portion.
%
%  -> Wikipedia citations, simply copy + paste the citation from the
%     menu on the LHS.
%
%
%- (02) Document Class and Options -------------------------------------------%
\documentclass[
%  pagesize,
  a4paper,
  pdftex,
%  fontsize=11pt,
  draft=false,
  twoside,
]{book}
%
%
%- (03) Standard package includes and options --------------------------------%
%\usepackage{draftwatermark} % draft watermark. Comment these 2 lines in final
%\SetWatermarkLightness{0.9} %
\usepackage{amsmath}       % amsmath & amssymb are almost ALWAYS required.
\usepackage{amssymb}       %
%
%\usepackage{verbatim}     % multiline commenting ( c++ equiv /* ... */ )
%
\usepackage{xcolor}         % pdflatex
\definecolor{neekiRed}{RGB}{172,40,41}
\definecolor{neekiBlue}{RGB}{62,70,157}
%
\usepackage{geometry}      % option for altering page dimensions if needed
\usepackage[pdftex]{graphicx} % including image files for figures (ie
                              % non-[E]PS)
%                          % valid types: jpeg, png, pdf
\usepackage{wrapfig}       % the figures themselves
\usepackage[numbers,
square,
longnamesfirst
]{natbib}                  % prettybib
\usepackage[pdftex]{hyperref} % clickable TOC and refs
%\usepackage[all]{xy}      % category theory helpers
%\xyoption{all}            % category theory helpers
%\input xy                 % category theory helpers
\usepackage{tikz}        % easy graphic thing
\usepackage{tabularx}    % easy tables
\usepackage{url}         % easy urls
\usepackage{multirow}    % 
\usepackage{lipsum}      % autogen placeholder text
%
%- (04) Custom Definitions and Alterations -----------------------------------%
\usepackage[T1]{fontenc} % font-doohickey
%\usepackage{tgadventor}  % font
%\usepackage[math]{iwona} % font
\usepackage[light,math]{kurier} % font
%
\linespread{1.5} % carneeki@ use approx 150% line spacing just like MaxDesign.
\hypersetup{
  % DO NOT CHANGE THESE
%   pdftitle={\metaTitle},
%   pdfauthor={\metaAuthorShort},
%   pdfsubject={\metaSubject},
%   pdfkeywords={\metaKeywords},
  pdfcreator={LaTeX},
  pdfproducer={LaTeX},
  pdftoolbar=false,
  % But change these to taste:
  pdffitwindow=false,   % window fit to page when opened
  pdfstartview={Fit},   % fits the width of the page to the window
                        % all (useful) opts: Fit, FitH, FitV,
  pdfnewwindow=true,    % open links in new window
  colorlinks=true,      % false = boxed links; true = colored links
  linkcolor=neekiRed, % color of internal links
  citecolor=neekiBlue, % color of links to bibliography
  filecolor=red, % color of file links
  urlcolor=red  % color of external links
}
%
%- (05) Custom Commands ------------------------------------------------------%
\newcommand{\derivative}[1][x]{\frac{\mathrm{d}}{\mathrm{d}#1}}
%
%- (06) Document Title and other metadata ------------------------------------%
%
\title{
  Grokking COMP125
}
%
\author{
  Adam J. Carmichael \\
  Undergraduate Student \\
  Department of Electronic   Engineering\\
  Macquarie University\\
  Sydney, Australia 2109\\
  Email: \url(adam.carmichael@ieee.org) \\
\and
  Carl E. Svensson \\
  PhD Candidate \\
  Department of Electronic   Engineering\\
  Macquarie University\\
  Sydney, Australia 2109\\
  Email: \url(carl.svensson@ieee.org) \\
} % author END Brace
%
%- (07) Start Document Content -----------------------------------------------%
\begin{document}
%- (07a) Misc Config ---------------------------------------------------------%
%-----------------------------------------------------------------------------%
%\cfoot{\thepage\ of \pageref{LastPage}} % page n of m
%
\maketitle
%-----------------------------------------------------------------------------%
%- Table of Contents ---------------------------------------------------------%
%-----------------------------------------------------------------------------%
\tableofcontents
%
\newpage
%
\chapter{COMP125 Preamble}
\label{sec:COMP125Preamble}
This unit studies programming as a systematic discipline and introduces more
formal software design methods. Programming skills are extended to include
elementary data structures and abstract data types. There is a strong emphasis
on problem solving and algorithms, including aspects of correctness, complexity
and computability.

\section{Learning outcomes}
It is expected that on completion of this unit/topic, students will:
\begin{itemize}
  \item demonstrate enhanced problem solving/algorithmic skills;
  \item be able to implement programs (from algorithms), showing an
        understanding of the underlying architecture of the computer;
  \item implement programs following standard software engineering practices (in
        particular document, test and debug).
\end{itemize}

\section{Assessments}
\begin{itemize}
  \item Build an MP3 player:
  \begin{itemize}
    \item Week 4: Simple Java Class: List of tracks (6\%)
    \item Week 8: Basic MP3 player, playlists, sound (12\%)
    \item Week 12: advanced player, GUI, wizbang (12\%)
  \end{itemize}
  \item Week 7: Mid-term test - in lecture (10\%)
  \item Tutorial Questions, practical problems (10\%)
  \item Final Exam (50\%)
\end{itemize}

Big changes from old COMP* units:
\begin{enumerate}
  \item Emphasis on professional software development (eg testing)
  \item Removed some material on low-level coding. (eg assembler)
  \item COMP115 - Processing
  \item COMP125 - Java (previously C++)
  \item COMP225 - Java from 2012 (previously C++)
  \item COMP229 - Can concentrate on software development (previously Java)
  \item COMP226 - Operating Systems / Computer Architecutre - May evolve to
        teach some C/C++
\end{enumerate}

%-----------------------------------------------------------------------------%
%- Introducing Java ----------------------------------------------------------%
%-----------------------------------------------------------------------------%
\chapter{Introducing Java}
\label{chap:IntroducingJava}

%-----------------------------------------------------------------------------%
%- Introducing Java :: A Section Here ----------------------------------------%
%-----------------------------------------------------------------------------%
\chapter{ASectionHere}
\label{sec:ASectionHere}
\lipsum[1]

%-----------------------------------------------------------------------------%
%- Developing and Testing Java Programs --------------------------------------%
%-----------------------------------------------------------------------------%
\chapter{Developing and Testing Java Programs}
\label{chap:DevelopingAndTestingJavaPrograms}
%-----------------------------------------------------------------------------%
%- Developing and Testing Java Programs :: A Section Here --------------------%
%-----------------------------------------------------------------------------%
\chapter{ASectionHere}
\label{sec:ASectionHere}
\lipsum[1]

%-----------------------------------------------------------------------------%
%- Object Oriented Design and Development ------------------------------------%
%-----------------------------------------------------------------------------%
\chapter{Object Oriented Design and Development}
\label{chap:ObjectOrientedDesignandDevelopment}
\lipsum[1]

%-----------------------------------------------------------------------------%
%- Search Algorithms ---------------------------------------------------------%
%-----------------------------------------------------------------------------%
\chapter{Search Algorithms}
\label{chap:SearchAlgorithms}
\lipsum[1]

%-----------------------------------------------------------------------------%
%- Software Development Topics -----------------------------------------------%
%-----------------------------------------------------------------------------%
\chapter{Software Development Topics}
\label{chap:SoftwareDevelopmentTopics}
\lipsum[1]

%-----------------------------------------------------------------------------%
%- Sorting Algorithms
%-----------------------------------------------------------------------------%
\chapter{Sorting Algorithms}
\label{chap:SortingAlgorithms}
\lipsum[1]

%-----------------------------------------------------------------------------%
%- Container Types
%-----------------------------------------------------------------------------%
\chapter{Container Types}
\label{chap:ContainerTypes}
\lipsum[1]

%-----------------------------------------------------------------------------%
%- Stacks and Queues
%-----------------------------------------------------------------------------%
\chapter{Stacks and Queues}
\label{chap:StacksAndQueues}
\lipsum[1]

%-----------------------------------------------------------------------------%
%- Implementing Linked Lists
%-----------------------------------------------------------------------------%
\chapter{Implementing Linked Lists}
\label{chap:ImplementingLinkedLists}
\lipsum[1]

%-----------------------------------------------------------------------------%
%- Recursive Algorithms and Data Structures
%-----------------------------------------------------------------------------%
\chapter{Recursive Algorithms and Data Structures}
\label{chap:RecursiveAlgorithmsAndDataStructures}
\lipsum[1]

%-----------------------------------------------------------------------------%
%- Inheritance and Polymorphism
%-----------------------------------------------------------------------------%
\chapter{Inheritance and Polymorphism}
\label{chap:InheritanceAndPolymorphism}
\lipsum[1]

%-----------------------------------------------------------------------------%
%- Software Development Topics
%-----------------------------------------------------------------------------%
\chapter{Software Development Topics}
\label{chap:SoftwareDevelopmentTopics}
\lipsum[1]

%-----------------------------------------------------------------------------%
%- Review
%-----------------------------------------------------------------------------%
\chapter{Review}
\label{chap:Review}
\lipsum[1]

%-----------------------------------------------------------------------------%
%- Acknowledgment ------------------------------------------------------------%
%-----------------------------------------------------------------------------%
\newpage
\chapter{Acknowledgements}
\label{sec:Acknowledgements}
I (Adam) had a whole swag of people to help me along the way. Listed, in no
particular order (because there is no fair way to list them), they are:
\begin{itemize}
  \item Carl Svensson, Macquarie University, for the \LaTeX, the maths, and the
  many late night sessions over a family dinner box, and the many in jokes and
  innuendoes\footnote{Giggity}.
  \item Michael Griffin, Macquarie University, for proof reading and finding
  errors.
  \item Josh Larietti, Macquarie University, for more maths.
  \item Celeste Cohen, for letting me show off stuff to her that I thought
  was pretty cool, whilst being completely irrelevant. For advice on page
  layout and wording. For being a friend when I needed one. For everything.
  \item The Heimlich Family, Macquarie University, for giving me a fantastic
  opportunity to put things I've learned into practice, and for the learning
  that resulted from it. To Mike, Luan, Sarah and Jaye for making FIRST happen
  in Australia, and for inviting me to become a part of it.
  \item FIRST Team 3132, The Thunder Down Under, for always holding me to high
  standards of Gracious Professionalism\texttrademark\footnote{Gracious
  Professionalism is a common law trademark of the United States Foundation for
  Inspiration and Recognition of Science and Technology (US FIRST).}
  \item Mark Leon, NASA, for the words of inspiration and wisdom when you
  spoke at the 2011 Honolulu FIRST FRC regionals. \quote{\ldots at the end of
  the day, it will be the engineers who save the world. This is why we do the
  math\ldots}
  \item Engineering \& math staff at Macquarie, David Wong, Yinan Kong, Sam
  Reisenfeld, Tony Parker, Rein Vaseilo, Barry McDonald. You make engineering
  awesome!
  \item It would be remiss of me to not mention the pit crew who make
  sure that I keep going lap after lap\ldots Nathan, Nick, Diana, Heidi, Hugh,
  Jessica, Frankie, VK2BV, Stephen VK2TQ, Will, Pippa, David, Emily, Andrew,
  John, Sue, Matthew, Richard, my brother Sean, and my mother and father.
\end{itemize}
% References
% Bibliography
\bibliography{COMP125}
\bibliographystyle{abbrvnat}
%
\end{document}